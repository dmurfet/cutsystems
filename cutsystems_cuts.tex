
%There is a bicategory $\cat{V}$ of smooth projective algebraic varieties over $\mathbb{C}$ \cite{??}, where the $1$-morphisms between varieties $\md{x}: A \lto B$ are Fourier-Mukai kernels, that is, complexes of coherent sheaves $\md{X}$ on the product $A \times B$. 

%For example, let $B = Proj(A)$ for some graded algebra $A = k[x_1,\ldots,x_n]/I$, and let $\md{X}, \md{Y}$ be complexes of vector bundles on $B$. In the bicategory $\cat{V}$ there are $1$-morphisms
%\[
%\xymatrix@C+2pc{
%\operatorname{pt} \ar[r]^-{\md{X}} & B \ar[r]^-{\md{Y}^{\vee}} & \operatorname{pt}
%}
%]
%where $\operatorname{pt} = \Spec(\mathbb{C})$. The composite in $\cat{V}$ is the complex
%\begin{equation}\label{eq:infinite_tensor_scheme}
%y^{\vee} \otimes^{\mathbb{L}} x = \md{Y}^{\vee} \otimes^{\mathbb{L}} \md{X} \cong 
%\end{equation}
%which is infinite-dimensional but has finite-dimensional cohomology. 

%Suppose that $\md{X}, \md{Y}$ are finitely described, for example that $X, Y$ are complexes of finitely generated graded free $A$-modules and $\md{X}, \md{Y}$ the associated complexes of sheaves, which are therefore described in terms of finite lists of matrices of homogeneous polynomials. 

\subsection{Splitting homotopies from connections}

We now develop the splitting homotopy of Example \ref{example:koszulsplit} in its most general form, in terms of Koszul complexes and connections. The splittings homotopies in this paper will all be derived as perturbations of these ``connection type'' examples. The material in the next two sections is from \cite{??}, we quote it here for the reader's convenience.

Let $k$ be a $\mathbb{Q}$-algebra and $R$ a $k$-algebra, $t_1,\ldots, t_n$ a quasi-regular sequence in $R$ such that $\bar{R} = R/(t_1,\ldots,t_n)$ is a finitely generated free $k$-module and $R$ admits a flat $k$-linear connection which is standard in the sense of \cite{??}
\[
\nabla: R \lto R \otimes_{k[t]} \Omega^1_{k[t]/k}
\]
and has the property that $\Ker(\nabla) + (t_1,\ldots,t_n) R = R$. With $\theta_i$ a formal variable of degree $-1$ the Koszul complex $K = K(t_1,\ldots,t_n)$ (see Definition \ref{defn:koszul}) has as its underlying graded module the exterior algebra on the $\theta_i$.

The connection $\nabla$ extends canonically to an operator on $R \otimes_{k[t]} \Omega^*_{k[t]/k}$ where $\Omega^* = \bigwedge \Omega^1$ is the exterior algebra. By identifying $d t_i$ with the generator $\theta_i$ of $K$ we may identify this tensor product with $K$ itself, so that $\nabla$ becomes a degree $-1$ $k$-linear operator on $K$. It is straightforward to check that the operator $[d_K, \nabla]$ is invertible on the graded submodule $K_{\le -1}$ of nonzero degree terms, and we define
\[
H = [d_K, \nabla]^{-1} \nabla\,.
\]

\begin{proposition} $H$ is a $k$-linear splitting homotopy on the complex $K$.
\end{proposition}
\begin{proof}
See \cite[Section 8.1]{papertoby}.
\end{proof}

The associated strong deformation retract of $\mathbb{Z}$-graded $k$-complexes is
\begin{equation}\label{eq:originalHdef}
\xymatrix@C+2pc{
(\bar{R},0) \ar@<-1ex>[r]_{\sigma} & (\Lambda, d_K), \ar@<-1ex>[l]_{\pi}
} \quad H
\end{equation}
where $\pi: K \lto \bar{R}$ is the canonical $R$-linear quasi-isomorphism which vanishes on $K_{\le -1}$ and is in degree zero the quotient map $R \lto \bar{R}$, and $\sigma$ is a $k$-linear embedding of $\bar{R}$ into $R$ uniquely determined by the choice of connection. 

\begin{remark} There is some flexibility to beginning with a connection rather than assuming that $R$ is a free $k[t]$-module, but this complication is not essential: the above is a very slight generalization of the simplest example, where $R = k[t]$ itself, $\nabla$ is the K\"ahler differential, and for a homogeneous polynomial $f \in k[t]$ and homogeneous tensor $\omega$
\[
[d_K, \nabla]^{-1}(f \omega) = \frac{1}{|f| + |\omega|} f \omega 
\]
and hence
\[
H( f \omega ) =  \frac{1}{|f| + |\omega|} \sum_i \partial_{t_i} (f) \theta_i \wedge \omega\,.
\]
In this case $\sigma$ is just the embedding of $k$ as the constant polynomials in $k[t]$.
\end{remark}

\subsection{Splitting homotopies on matrix factorisations}

With the notation of the previous section let $(X,d_X)$ be a finite rank matrix factorisation over $R$ of some $W \in k$. That is, $X$ is a finite rank free $R$-module and $d_X$ is $R$-linear, and the potential $W$ is in the image of the structure map $k \lto R$. The aim is to present $X \otimes K$ as being $k$-homotopy equivalent to a finite rank matrix factorisation over $k$, using a splitting homotopy on $X \otimes K = (X \otimes \Lambda, d_X + d_K)$ obtained via the perturbation lemma from a splitting homotopy on $(X \otimes \Lambda, d_K)$, viewing $d_X$ as a perturbation on the latter.

Choosing a homogeneous $R$-basis allows us to write $X \cong V \otimes_k R$ for some $\mathbb{Z}_2$-graded free $k$-module $V$, in which case
\begin{equation}\label{eq:lambdaiso2}
X \otimes \Lambda \cong (V \otimes_k R) \otimes_R \Lambda \cong V \otimes_k \Lambda
\end{equation}
In this way, using the basis, the $k$-linear splitting homotopy $H$ on $K = (\Lambda, d_K)$ gives rise to a $k$-linear operator $1 \otimes H$ on $X \otimes \Lambda$, which we again denote by $H$.

\begin{lemma} $H$ is a $k$-linear splitting homotopy on the complex $(X \otimes \Lambda, d_K)$.
\end{lemma}

The associated strong deformation retract is
\[
\xymatrix@C+2pc{
(\bar{X},0) \ar@<-1ex>[r]_-{\sigma} & (X \otimes \Lambda, d_K), \ar@<-1ex>[l]_-{\pi}
} \quad H
\]
where $\bar{X} = X / (t_1,\ldots,t_n)X = X \otimes_R \bar{R}$, $\pi$ is $R$-linear induced by the map $\pi: K \lto \bar{R}$, and $\sigma = 1 \otimes \sigma$ is defined using map $\sigma$ of \eqref{eq:originalHdef} and the isomorphism \eqref{eq:lambdaiso2}.

We now view $\tau = d_X$ as a perturbation of $(X \otimes \Lambda, d_K)$. Naturally $(d_K + \tau)^2 = W$, so that the potential is changed by this perturbation. The transference problem asks how to define a splitting homotopy $\phi_\infty$ on the matrix factorisation $X \otimes K = (X \otimes \Lambda, d_K + d_X)$ in such a way that the underlying graded module is $\bar{X}$. Since $H d_X$ has finite order, it follows from the perturbation lemma (Theorem \ref{theorem:pertlemma}) that
\[
\phi_\infty = \sum_{m \ge 0} (-1)^m (H d_X)^m H
\]
is a $k$-linear splitting homotopy achieving the desired aim.

The next result is \cite[Proposition 7.1]{??}.

\begin{corollary} $\phi_\infty$ is a $k$-linear splitting homotopy on $X \otimes K$ and the associated $k$-linear strong deformation retract of linear factorisations of $W$ is of the form
\begin{equation}
\xymatrix@C+2pc{
(\bar{X},d_X) \ar@<-1ex>[r]_-{\sigma_\infty} & (X \otimes \Lambda, d_X + d_K), \ar@<-1ex>[l]_-{\pi}
} \quad \phi_\infty
\end{equation}
where $A = d_X( 1 + H d_X )^{-1}$ and $\sigma_\infty = \sigma - H A \sigma$.
\end{corollary} 
\begin{proof}
We need only justify why in \eqref{eq:perturbedsdr} we have $d_\infty = d_X$ and $f_\infty = \pi$. But this follows from $H \sigma = 0$ and $\pi H = 0$.
\end{proof}
