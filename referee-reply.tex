\documentclass[english,letter paper,12pt,leqno]{article}
\usepackage{stmaryrd}
\usepackage{amsmath, amscd, amssymb, mathrsfs, accents, amsfonts,amsthm}
\usepackage[all]{xy}
\usepackage{dsfont}
\usepackage{tikz}
\def\nicedashedcolourscheme{\shadedraw[top color=blue!22, bottom color=blue!22, draw=gray, dashed]}
\def\nicecolourscheme{\shadedraw[top color=blue!22, bottom color=blue!22, draw=white]}
\def\nicepalecolourscheme{\shadedraw[top color=blue!12, bottom color=blue!12, draw=white]}
\def\nicenocolourscheme{\shadedraw[top color=gray!2, bottom color=gray!25, draw=white]}
\def\nicereallynocolourscheme{\shadedraw[top color=white!2, bottom color=white!25, draw=white]}
\definecolor{Myblue}{rgb}{0,0,0.6}
\usepackage[a4paper,colorlinks,citecolor=Myblue,linkcolor=Myblue,urlcolor=Myblue,pdfpagemode=None]{hyperref}

\SelectTips{cm}{}

\setlength{\evensidemargin}{0.1in}
\setlength{\oddsidemargin}{0.1in}
\setlength{\textwidth}{6.3in}
\setlength{\topmargin}{0.0in}
\setlength{\textheight}{8.5in}
\setlength{\headheight}{0in}

\newtheorem{theorem}{Theorem}[section]
\newtheorem{proposition}[theorem]{Proposition}
\newtheorem{lemma}[theorem]{Lemma}
\newtheorem{corollary}[theorem]{Corollary}
\newtheorem{setup}[theorem]{Setup}

\newtheoremstyle{example}{\topsep}{\topsep}
	{}
	{}
	{\bfseries}
	{.}
	{2pt}
	{\thmname{#1}\thmnumber{ #2}\thmnote{ #3}}
	
	\theoremstyle{example}
	\newtheorem{definition}[theorem]{Definition}
	\newtheorem{example}[theorem]{Example}
	\newtheorem{remark}[theorem]{Remark}
	\newtheorem{strat}[theorem]{Strategy}

\numberwithin{equation}{section}

% Operators
\def\eval{\operatorname{ev}}
\def\res{\operatorname{Res}}
\def\Coker{\operatorname{Coker}}
\def\Ker{\operatorname{Ker}}
\def\im{\operatorname{Im}}
\def\can{\operatorname{can}}
\def\K{\mathbf{K}}
\def\D{\mathbf{D}}
\def\N{\mathbf{N}}
\def\LG{\mathcal{LG}}
\def\Ab{\operatorname{Ab}}
\def\Hom{\operatorname{Hom}}
\def\modd{\operatorname{mod}}
\def\Modd{\operatorname{Mod}}
\def\be{\begin{equation}}
\def\ee{\end{equation}}
\def\nN{\mathds{N}}
\def\nZ{\mathds{Z}}
\def\nQ{\mathds{Q}}
\def\nR{\mathds{R}}
\def\nC{\mathds{C}}
\def\L{\mathcal{C}}
\def\ferm{\gamma}
\def\fermc{\gamma^\dagger}
\DeclareMathOperator{\Ext}{Ext}
\DeclareMathOperator{\Tr}{Tr}
\DeclareMathOperator{\End}{End}
\DeclareMathOperator{\rank}{rank}
\DeclareMathOperator{\tot}{Tot}
\DeclareMathOperator{\ch}{ch}
\DeclareMathOperator{\str}{str}
\DeclareMathOperator{\hmf}{hmf}
\DeclareMathOperator{\HMF}{HMF}
\DeclareMathOperator{\hf}{HF}
\DeclareMathOperator{\At}{At}
\DeclareMathOperator{\Cat}{Cat}
\DeclareMathOperator{\Spec}{Spec}

\begin{document}

% Commands
\def\Res{\res\!}
\newcommand{\ud}{\mathrm{d}}
\newcommand{\Ress}[1]{\res_{#1}\!}
\newcommand{\cat}[1]{\mathcal{#1}}
\newcommand{\lto}{\longrightarrow}
\newcommand{\xlto}[1]{\stackrel{#1}\lto}
\newcommand{\mf}[1]{\mathfrak{#1}}
\newcommand{\md}[1]{\mathscr{#1}}
\def\sus{\l}
\def\l{\,|\,}
\def\sgn{\textup{sgn}}

\title{Reply to referee report}

I thank the referee for their useful comments, and make the following remarks in reply:

\begin{enumerate}
\item The referee is right to observe that while the term ``cut system'' appears prominently in the title and elsewhere in the paper, it is never really defined. After some consideration it seems to me better to remove the term from the paper altogether, and leave the development to sequels where it belongs. To this end I have removed all occurrences of ``cut system'' and referred instead to the ``cut operation'' which (I hope) is clearly defined in the text. Other occurrences of ``cut system'' (for instance in the discussion of linear logic) have been replaced by ``enrichment'' which is a more useful descriptor.

\item p.3 added reference for Atiyah classes.

\item p.5 the logical variable has been renamed to $\alpha$ for clarity.

\item p.8 the $\Psi_a$ are not given a form of $2$-functoriality in $a$, so no.

\item p.10 added a more precise reference

\item p.11 the nonstandard notation has been fixed.

\item p.13 Added $F = k^n$.

\item p.14 added a definition for $\cat{M}$.

\item Typos in the penultimate sentence of the proof of Lemma 2.27 (now Lemma 2.28) corrected.

\item Extra line in the proof of Lemma 3.2 fixed.

\item Clarified the matrix assigned to $1 \in k$ in Remark 3.3.

\item In (3.5) explained the notation for K\"ahler differentials

\item Corrected ``this odd''

\item In Lemma 3.9 clarified which differential is being referred to.

\item Removed the reference to linear factorisation (it is not necessary).

\item Fixed the definition of a morphism of strong deformation retracts.

\item Changed one of the two $A$s to a $C$, near (4.9).

\item Fixed equality that should have been $\simeq$ in second sentence of Section $4.2$.

\item Proof of Lemma 4.16, changed ``commute'' to ``anticommute''.

\item Removed Remark 4.30.

\item Fixed typo in Definition 4.39.

\item End of p.44 added reference to definition of $\cat{C}_A$.

\item Fixed the definition of $T$ in Appendix B, and the typo in (B.7).

\item Fixed references.

\end{enumerate}

In addition I have made some minor typographical corrections, and made the following more serious corrections:

\begin{enumerate}
\item I actually need the base ring $k$ to be noetherian in order for some completions to be flat algebras (see the footnote to the first sentence of the Background section), but this wasn't being assumed in the original version. 
\item In the original I took a completion from $k[y]$ to $k\llbracket y \rrbracket$ in order to ensure the existence of a connection (p.17 of the original). However what is actually needed is that this completion is separated and complete for the adic topology determined by the ideal $I$ generated by the partial derivatives $\partial_{y_i} V$, which is not true unless the origin is the only critical point. This was therefore a genuine error in the paper. However it is easily fixed by changing the completion to the $I$-adic completion everywhere, which has been done (see Definition 3.5).
\end{enumerate}

\end{document}