\documentclass[english,letter paper,12pt,leqno]{article}
\usepackage{stmaryrd}
\usepackage{amsmath, amscd, amssymb, mathrsfs, accents, amsfonts,amsthm}
\usepackage[all]{xy}
\usepackage{tikz}
\def\nicedashedcolourscheme{\shadedraw[top color=blue!22, bottom color=blue!22, draw=gray, dashed]}
\def\nicecolourscheme{\shadedraw[top color=blue!22, bottom color=blue!22, draw=white]}
\def\nicepalecolourscheme{\shadedraw[top color=blue!12, bottom color=blue!12, draw=white]}
\def\nicenocolourscheme{\shadedraw[top color=gray!2, bottom color=gray!25, draw=white]}
\def\nicereallynocolourscheme{\shadedraw[top color=white!2, bottom color=white!25, draw=white]}
\definecolor{Myblue}{rgb}{0,0,0.6}
\usepackage[a4paper,colorlinks,citecolor=Myblue,linkcolor=Myblue,urlcolor=Myblue,pdfpagemode=None]{hyperref}

\SelectTips{cm}{}

\setlength{\evensidemargin}{0.1in}
\setlength{\oddsidemargin}{0.1in}
\setlength{\textwidth}{6.3in}
\setlength{\topmargin}{0.0in}
\setlength{\textheight}{8.5in}
\setlength{\headheight}{0in}

\newtheorem{theorem}{Theorem}[section]
\newtheorem{proposition}[theorem]{Proposition}
\newtheorem{lemma}[theorem]{Lemma}
\newtheorem{corollary}[theorem]{Corollary}

\newtheoremstyle{example}{\topsep}{\topsep}
	{}
	{}
	{\bfseries}
	{.}
	{2pt}
	{\thmname{#1}\thmnumber{ #2}\thmnote{ #3}}
	
	\theoremstyle{example}
	\newtheorem{definition}[theorem]{Definition}
	\newtheorem{example}[theorem]{Example}
	\newtheorem{remark}[theorem]{Remark}

\numberwithin{equation}{section}

% Operators
\def\eval{\operatorname{ev}}
\def\res{\operatorname{Res}}
\def\Coker{\operatorname{Coker}}
\def\Ker{\operatorname{Ker}}
\def\im{\operatorname{Im}}
\def\can{\operatorname{can}}
\def\K{\mathbf{K}}
\def\D{\mathbf{D}}
\def\N{\mathbf{N}}
\def\LG{\mathcal{LG}}
\def\Ab{\operatorname{Ab}}
\def\Hom{\operatorname{Hom}}
\def\modd{\operatorname{mod}}
\def\Modd{\operatorname{Mod}}
\DeclareMathOperator{\Ext}{Ext}
\DeclareMathOperator{\Tr}{Tr}
\DeclareMathOperator{\End}{End}
\DeclareMathOperator{\rank}{rank}
\DeclareMathOperator{\tot}{Tot}
\DeclareMathOperator{\ch}{ch}
\DeclareMathOperator{\str}{str}
\DeclareMathOperator{\hmf}{hmf}
\DeclareMathOperator{\HMF}{HMF}
\DeclareMathOperator{\hf}{HF}
\DeclareMathOperator{\At}{At}
\DeclareMathOperator{\Cat}{Cat}
\DeclareMathOperator{\Spec}{Spec}

\begin{document}

% Commands
\def\Res{\res\!}
\newcommand{\ud}{\mathrm{d}}
\newcommand{\Ress}[1]{\res_{#1}\!}
\newcommand{\cat}[1]{\mathcal{#1}}
\newcommand{\lto}{\longrightarrow}
\newcommand{\xlto}[1]{\stackrel{#1}\lto}
\newcommand{\mf}[1]{\mathfrak{#1}}
\newcommand{\md}[1]{\mathscr{#1}}
\def\l{\,|\,}
\def\sgn{\textup{sgn}}

\title{Computing with cut systems}
\author{Daniel Murfet}

\maketitle

\begin{abstract}
A cut system has objects, $1$-morphisms and $2$-morphisms, similar to a bicategory. In contrast to the composition of $1$-morphisms in a bicategory, the cut operation on $1$-morphims in a cut system produces representations of Clifford algebras in supercategories rather than bare objects. The motivating example of a cut system refines the bicategory of Landau-Ginzburg models, in which the objects are hypersurface singularities and the $1$-morphisms are matrix factorisations.
\end{abstract}

\section{Introduction}

In this paper we develop a new semantics of composition using a bicategorical formalism that we call \emph{cut systems}, inspired by Gentzen's cut-elimination \cite{gentzen} and Girard's geometry of interaction \cite{girard_towards}. The formalism is used to ``make computable'' a bicategory of isolated hypersurface singularities and matrix factorisations that has been studied in the context of topological field theory \cite{McNameethesis, cr0909.4381}. This means that the operations making up the data of the bicategory are given algorithmically, and can be implemented on a computer. One of the main applications, to be given in a sequel \cite{cutsystems2}, is to construct a model of a fragment of Girard's linear logic \cite{girard_llogic}. Roughly speaking, this provides mathematical semantics for a programming language in which data types are interpreted as hypersurface singularities, and programs are interpreted as homotopy equivalence classes of matrix factorisations.

What is a semantics of composition? Consider that in a category or bicategory $\cat{A}$, for every triple of objects $a,b,c$ and morphisms
\[
\xymatrix@C+2pc{
a \ar[r]^-{\Phi} & b \ar[r]^-{\Psi} & c
}
\]
there is an assignment to $(\Psi, \Phi)$ of a composition $\Psi \circ \Phi$. A \emph{semantics of composition} for $\cat{A}$ is a mathematical construct providing some kind of inner structure for this ``assignment'' which is coherent as $a,b,c,\Phi,\Psi$ vary. The main examples of this kind of semantics are in computer science and mathematical logic. 

For example, suppose that $\cat{A}$ is the category of sets and that we have functions
\[
\xymatrix@C+2pc{
\mathbb{N} \ar[r]^-{\Phi} & \mathbb{N} \ar[r]^-{\Psi} & \mathbb{N}
}
\]
where $\Phi$ and $\Psi$ are ``finitely described'' in the sense that they are computed by algorithms $X$ and $Y$, respectively. To be explicit, let us take $X,Y$ to be terms in the $\lambda$-calculus \cite{selinger}. If $[-]$ denotes assignment to an algorithm of the function it describes, then $\Phi = [X], \Psi = [Y]$ and $\Psi \circ \Phi = [Y \l X]$ where $Y \l X$ is the result of feeding the output of the algorithm $X$ as an input to the algorithm $Y$. In the logic literature this concatenation is known as a \emph{cut}.\footnote{We note that the notation $Y \l X$ for the cut is not standard, so don't blame the logicians for it.} To encounter the desired semantics we need the additional observation that the cut $Y \l X$ is not in normal form (whatever this means) but it may be reduced to normal form by a rewriting process on terms in the $\lambda$-calculus called $\beta$-reduction. This does not change the associated function: the normal form of $Y \l X$ also computes $\Psi \circ \Phi$. 

A semantics of composition for the category of computable functions is then given by the process which begins with the pair $(Y,X)$ of finite descriptions of $\Psi,\Phi$, forms the cut $Y \l X$, and then computes a normal form giving a finite description of $\Psi \circ \Phi$.

\medskip

This seems very far from geometry, but we can informally recast a problem of algebraic geometry in a similar light. Suppose $A,B,C$ are smooth projective algebraic varieties and that we are given triangulated functors between derived categories of coherent sheaves
\[
\xymatrix{
\mathbf{D}^b(A) \ar[r]^-{\Phi} & \mathbf{D}^b(B) \ar[r]^-{\Psi} & \mathbf{D}^b(C)
}\,.
\]
Suppose that $\Phi,\Psi$ are finitely described, in the sense that they are both integral functors associated to kernels: $\Phi$ is convolution with a complex of coherent sheaves $X$ on $A \times B$, and $\Psi$ is convolution with a complex $Y$ on $B \times C$. If for example $A,B,C$ are the $\operatorname{Proj}$ of graded algebras, $X$ and $Y$ can be described by lists of matrices of homogeneous polynomials. An obvious question is: what is the finite description of the functor $\Psi \circ \Phi$?

It is easy to see that $\Psi \circ \Phi$ is convolution with the kernel $Y \otimes^{\mathbb{L}}_{\cat{O}_B} X$, which is a complex of quasi-coherent sheaves on $A \times C$ with coherent cohomology, but this is not described by a finite amount of data. Finding a finite description of $\Psi \circ \Phi$ is precisely the problem of determing the complex of coherent sheaves which is isomorphic, in $\mathbf{D}^b(A \times C)$, to $Y \otimes^{\mathbb{L}}_{\cat{O}_B} X$. We refer to such a complex of coherent sheaves as a \emph{finite model} of $Y \otimes^{\mathbb{L}}_{\cat{O}_B} X$.

Consider the case where $A = C = \operatorname{Spec}(\mathbb{C})$, so $X,Y$ are complexes of coherent sheaves on $B$ and $Y \otimes^{\mathbb{L}}_{\cat{O}_B} X \cong \mathbb{R}\! \operatorname{Hom}_{\cat{O}_B}(Y^{\vee}, X)$ is a complex of infinite-dimensional vector spaces with finite-dimensional cohomology. An example of a finite model is an isomorphism in the derived category of all $\mathbb{C}$-vector spaces of this complex with its cohomology
\begin{equation}\label{eq:computing_hom_derived}
Y \otimes^{\mathbb{L}}_{\cat{O}_B} X \cong \bigoplus_{n \in \mathbb{Z}} \Ext^n(Y^{\vee},X)[-n]\,.
\end{equation}
One finite description of $Y \otimes^{\mathbb{L}}_{\cat{O}_B} X$, and transitively of $\Psi \circ \Phi$, is therefore a list of numbers: the dimensions of the vector spaces on the right hand side of \eqref{eq:computing_hom_derived}. Determining these numbers is difficult, to put it mildly: a fundamental problem in algebraic geometry is to understand how these numbers depend on the input data $X$ and $Y$.

By a (perhaps naive) analogy with computable functions, one might wonder if in the general case there exists an object $Y \l X$ in some triangulated category together with a reduction process that would help find a finite description for $\Psi \circ \Phi$. This is the question of whether or not there is a semantics of composition in the bicategory of smooth projective varieties and Fourier-Mukai kernels \cite{ct1007.2679}. Unfortunately, we do not know such a theory!

However, it turns out that there \emph{is} a satisfactory semantics of this kind when instead of smooth projective varieties we consider isolated hypersurface singularities $W,V,U$ and triangulated functors between homotopy categories of matrix factorisations \cite{yoshino98}
\begin{equation}\label{eq:pair_functors_hmf}
\xymatrix{
\hmf(W) \ar[r]^-{\Phi} & \hmf(V) \ar[r]^-{\Psi} & \hmf(U)
}\,.
\end{equation}
To explain further, we need to introduce the language of cut systems.

\medskip

A \emph{cut system} $\cat{A}$ over a $\mathbb{Q}$-algebra $k$ has a class of objects and for each pair $a,b$ of objects a $k$-linear supercategory $\cat{A}(a,b)$. Informally, one should think of this as a category of finite models. There is an operation on $\cat{A}$, the \emph{cut}, which assigns to $X \in \cat{A}(a,b)$ and $Y \in \cat{A}(b,c)$ a representation $Y \l X$ in $\cat{A}(a,c)$ of a Clifford algebra which depends on the intermediate object $b$. The object $Y \l X$ and the endomorphisms giving the Clifford action are required to be effectively computable, say by explicit formulas, from the input data $X,Y$. In the example of a cut system developed in this paper, Clifford algebras arise as endomorphism algebras of exterior algebras, which in turn arise as the underlying graded modules of Koszul complexes. Associated to the cut system $\cat{A}$ is a bicategory $\cat{A}^\bullet$ in which the composition is derived from the cut operation, and so is also effectively computable.

A \emph{cut model} of a bicategory $\cat{B}$ is a cut system $\cat{A}$ with the same objects as $\cat{B}$, together with an equivalence $I: \cat{B} \lto \cat{A}^\bullet$. There are fully faithful functors $I: \cat{B}(a,b) \lto \cat{A}(a,b)$ which we think of as assigning finite models to the $1$-morphisms of $\cat{B}$, and an isomorphism of Clifford representations for every composable pair of $1$-morphisms $X,Y$ in $\cat{B}$, where $S$ denotes the spinor representation:
\begin{equation}\label{eq:intro_cut_semantics}
I(Y) \l I(X) \cong S \otimes_k I( Y \circ X )\,.
\end{equation}
The cut model provides $\cat{B}$ with a semantics of composition: given finite models $I(X), I(Y)$ of $X,Y$ there is a process beginning with the cut $I(Y) \l I(X)$ which extracts the subobject $I(Y \circ X)$ using the Clifford action. The result $I( Y \circ X )$ of this computation is the desired finite model of the composition $Y \circ X$ in $\cat{B}$.
\medskip

The formalism is adapted to, and motivated by, an example: the bicategory of Landau-Ginzburg models $\LG$. Let us return to the diagram \eqref{eq:pair_functors_hmf} as our starting point. There is an analogue of integral functors in this setting, so that it makes sense to talk about $\Phi$ and $\Psi$ as being respectively given by convolution with kernels $X,Y$. If $W \in k[x_1,\ldots,x_n]$ and $V \in k[z_1,\ldots,z_m]$ define isolated hypersurface singularities then $X$ is a finite rank matrix factorisation of $V - W$ over $k[x,z]$, and similarly for $Y$.

Again, the question is: what is the finite description of $\Psi \circ \Phi$? It is naturally given by convolution with a matrix factorisation $Y \otimes X$ which is not finitely generated: finding a finite description of $\Psi \circ \Phi$ therefore amounts to finding a finite model for $Y \otimes X$. To fit this into the formalism of cut systems, we use the  bicategory $\LG$ in which the objects are isolated hypersurface singularities\footnote{Roughly speaking. See Section \ref{section:superbicatLG}.} and $1$-morphisms $W \lto V$ are matrix factorisations of $V - W$. In this bicategory, the finite descriptions of the functors in \eqref{eq:pair_functors_hmf} give rise to a composable pair of $1$-morphisms
\begin{equation}\label{eq:cut_lg}
\xymatrix@C+2pc{
W \ar[r]^-{X} & V \ar[r]^-{Y} & U
}\,.
\end{equation}
whose composition in $\LG$ is the tensor product $Y \otimes X$.

The main results of this paper are the construction of a cut sytem $\cat{L}$ and cut model $I: \LG \lto \cat{L}^\bullet$. In the situation of \eqref{eq:cut_lg} $X$ and $Y$ are their own finite model (so $I(X) = X$ and $I(Y) = Y$) and a finite model of the composite $Y \otimes X$ in $\cat{B}$ may be computed from the cut $Y \l X$ in $\cat{L}$, which is an explicitly given finite rank matrix factorisation of $U - W$ together with a Clifford action described in terms of derivatives of the differentials of $X,Y$ and certain invariants called Atiyah classes.

One consequence is that $\LG$ is equivalent to the bicategory $\cat{L}^\bullet$, which is \emph{computable} in the sense that its operations may be embodied in software. To some extent this software already exists: a component of what we are here calling the cut system $\cat{L}$ (developed earlier in joint work with Dyckerhoff \cite{dm1102.2957}\footnote{For a more detailed discussion of the relation of this work with the present paper, see Remark \ref{remark:relation_to_toby_paper}.}) has already been implemented in the computer algebra package Singular in joint work with Carqueville \cite{khovhompaper}, and used to compute tables of Khovanov-Rozansky knot homologies. The code was also used in the discovery of the generalised orbifolding relations among ADE singularities by Carqueville, Ros Camacho and Runkel \cite{ade}. 

\medskip
In conclusion, the purpose of this paper is to develop the theory of cut systems as a mathematical foundation for the application to be given in \cite{cutsystems2}. Let us briefly summarise the contents of that work: one can show that the superbicategory $\cat{L}^\bullet$ is monoidal, has adjoints and is pivotal. This makes it a natural target for models of Girard's linear logic, which is a generalisation of the $\lambda$-calculus in which the terms may be viewed as algorithms that have a refined syntax of resource usage compared to the $\lambda$-calculus. The model of linear logic to be given in \cite{cutsystems2} interprets algorithms $x,y$ as $1$-morphisms $[x],[y]$ in $\cat{L}^\bullet$. 

This interpretation has two novelties: first of all, the target $\cat{L}^\bullet$ has a semantics of composition that may be compared to cut-elimination in linear logic (the analogue of $\beta$-reduction in the $\lambda$-calculus) as in Girard's geometry of interaction \cite{girard_towards}. This connection is what motivates the terminology of ``cut systems'' in the present paper. The second novelty is that, compared to the usual mathematical semantics of programming languages where algorithms are interpreted as functions between sets, our interpretation is bicategorical: in particular, equations $[x] \circ [y] \cong [y] \circ [x]$ or $[x]^2 \cong [x]$ on programs are represented by data that is part of the semantics.

%. For a precise definition see Section \ref{??}}. To such a $W$ we associate a category of algebraic objects called matrix factorisations. A matrix factorisation of $W(x)$ is a matrix of polynomials $D$ satisfying $D^2 = W \cdot I_n$, where $I_n$ is the identity matrix. 

%For $W = x y^2 + y^3$ an example is
%\[
%D_1 = \begin{pmatrix} 0 & x - i y \\ x + iy & 0 \end{pmatrix} %, \qquad D_2 = \begin{pmatrix} 0 & y - x \\ (y - e^{2 \pi i /3} x)(y - e^{4 \pi i /3} x ) & 0 \end{pmatrix}
%\]
%A \emph{morphism} of matrix factorisations $\Phi: D_1 \lto D_2$ is a matrix of polynomials $\Phi$ satisfying the relation $\Phi D_1 = D_2 \Phi$. Together the matrix factorisations of $W$ and the morphisms between them form the category of matrix factorisations of $W$.


%In the second part of the paper we explain how the multiplicative fragment of linear logic can be interpreted in the bicategory of Landau-Ginzburg models, with formulas mapped to isolated hypersurface singularities and proofs mapped to matrix factorisations. The multiplicative fragment is not very expressive, so our focus in the present paper is not on embedding sophisticated programs into $\cat{LG}$. Rather, we explain how cut systems give a dynamic model of cut-elimination following Girard's geometry of interaction. The basic construction of a cut system involves homological perturbation, and by interpreting this as a fixed point construction we will show that the result of normalization can be expressed in a power series expansion similar to Girard's famous execution formula.

\medskip

The paper is structured as follows: in Section \ref{section:background} we recall the definition of superbicategories and Clifford algebras, and introduce the Clifford thickening of a supercategory. In Section \ref{section:cut_systems} we define cut systems and the superbicategory associated to a cut system, then we construct the cut model of $\LG$ in Section \ref{section:lg_cut_system}.

\medskip

\emph{Acknowledgements.} Thanks to Nils Carqueville and Ingo Runkel for helpful comments on the draft. Although the sequel on linear logic is the proper place to fully acknowledge the debt, we cannot neglect to mention the important role that the beautiful ideas in Girard's geometry of interaction program \cite{girard_towards} had in shaping the present work.

\section{Background}\label{section:background}

Throughout $k$ is a commutative $\mathbb{Q}$-algebra. By default categories and functors are $k$-linear.

\subsection{Supercategories}

A superbicategory has for every $1$-morphism $F$ an associated $1$-morphism $\Psi F$ with $\Psi^2 \cong 1$, and in this sense the categories of $1$-morphisms are $\mathbb{Z}_2$-graded. The bicategory of Landau-Ginzburg models is an example of a superbicategory. Since Clifford algebras (which are $\mathbb{Z}_2$-graded) play a fundamental role in the definition of cut systems, superbicategories are the natural language. Our main references for the foundations are \cite{ellis_lauda,kang,kang2}, but we give the full definitions below as these references work only with strict bicategories.

\begin{definition} A \emph{supercategory} is a category $\cat{C}$ together with a functor $\Psi: \cat{C} \lto \cat{C}$ and a natural isomorphism $\xi: \Psi^2 \lto 1_{\cat{C}}$ satisfying the condition
\[
\xi * 1_{\Psi} = 1_{\Psi} * \xi
\]
as natural transformations $\Psi^3 \lto \Psi$. A \emph{superfunctor} $(F, \gamma)$ from a supercategory $(\cat{C}, \Psi_{\cat{C}})$ to a supercategory $(\cat{D}, \Psi_{\cat{D}})$ is a functor $F: \cat{C} \lto \cat{D}$ together with a natural isomorphism $\gamma: F \Psi_{\cat{C}} \lto \Psi_{\cat{D}} F$ satisfying
\[
1_F * \xi = (\xi * 1_F ) ( 1_{\Psi} * \gamma ) ( \gamma * 1_{\Psi} )\,.
\]
A \emph{supernatural transformation} $\varphi: F \lto G$ between superfunctors $(F,\gamma_F), (G,\gamma_G)$ is a natural transformation making the following diagram commute:
\[
\xymatrix{
F \Psi \ar[r]^-{\varphi \Psi}\ar[d]_-{\gamma_F} & G \Psi \ar[d]^-{\gamma_G}\\
\Psi F \ar[r]_-{\Psi \varphi} & \Psi G
}\,.
\] 
\end{definition}

\begin{example}\label{example:Aassup} If $A$ is a $\mathbb{Z}_2$-graded $k$-algebra there is a supercategory $\cat{A}$ with set of objects $|\cat{A}| = \mathbb{Z}_2$ and morphisms $\cat{A}(0,0) = \cat{A}(1,1) = A_0, \cat{A}(0,1) = \cat{A}(1,0) = A_1$ with the obvious composition and addition rules. The functor $\Psi$ is defined on objects by $\Psi(i) = i+1$ and on morphisms by the identity, while $\xi = 1$.
\end{example}

Let $\cat{C}$ be a supercategory. For objects $X,Y \in \cat{C}$ we define the $\mathbb{Z}_2$-graded $k$-module
\[
\cat{C}^*(X,Y) = \prod_{i \in \mathbb{Z}_2} \cat{C}(X, \Psi^i Y)\,.
\]
There is for any triple $X,Y,Z$ of objects a morphism of graded modules
\[
\cat{C}^*(Y,Z) \otimes_k \cat{C}^*(X, Y) \lto \cat{C}^*(X,Z)
\]
given for instance by $\varphi \otimes \psi \mapsto \xi \Psi( \varphi ) \psi$ when $\varphi, \psi$ are both of degree one. This makes $\cat{C}$ into a category enriched over the monoidal category of $\mathbb{Z}_2$-graded $k$-modules.

There are isomorphisms of graded modules $\cat{C}^*(X, \Psi Y) \cong \Psi \cat{C}^*(X, Y) \cong \cat{C}^*(\Psi X, Y)$ where $\Psi V = V [1]$ denotes the grading shift on a $\mathbb{Z}_2$-graded $k$-module. For $\mathbb{Z}_2$-graded $k$-modules $V,W$ we write $\Hom^0_k(V, W)$ for the module of degree zero maps and $\Hom_k^*(V, W)$ for the $\mathbb{Z}_2$-graded module of homogeneous maps. If $A$ is a $\mathbb{Z}_2$-graded $k$-algebra then $A^{\operatorname{op}}$ denotes the same underlying graded module with multiplication $f * g = (-1)^{|f||g|} gf$.

\begin{definition} Let $A$ be a $\mathbb{Z}_2$-graded $k$-algebra and $\cat{C}$ a supercategory. A \emph{left $A$-module in $\cat{C}$} is an object $X \in \cat{C}$ together with a morphism of $\mathbb{Z}_2$-graded algebras $A \lto \cat{C}^*(X,X)$. A \emph{morphism} of left $A$-modules is a morphism in $\cat{C}$ which commutes with the $A$-action in the obvious sense. The category of left $A$-modules in $\cat{C}$ is denoted $\cat{C}_A$. Right modules are defined similarly, using $A^{\operatorname{op}}$.
\end{definition}

\begin{remark}\label{remark:supercat_idempcomp} Viewing $A$ as a supercategory $\cat{A}$ as in Example \ref{example:Aassup}, there is an equivalence between the category of $A$-modules in $\cat{C}$ and the category of superfunctors $\cat{A} \lto \cat{C}$ and supernatural transformations.
\end{remark}

\begin{remark}\label{remark:idempotent_completion} The \emph{idempotent completion} $\cat{C}^\omega$ of a category $\cat{C}$ has as objects pairs $(X,e)$ consisting of an object $X \in \cat{C}$ and an idempotent endomorphism $e: X \lto X$. In $\cat{C}^\omega$ a morphism $\phi: (X,e) \lto (Y,e)$ is a morphism $\phi: X \lto Y$ satisfying $\phi e = \phi$ and $e \phi = \phi$. The identity on $(X,e)$ is $e$. There is a fully faithful functor $\iota: \cat{C} \lto \cat{C}^{\omega}, \iota(X) = (X,1_X)$ which is an equivalence if $\cat{C}$ is idempotent complete (i.e. all idempotents split in $\cat{C}$).

If $\cat{C}$ is a supercategory then there is a functor $\Psi^\omega: \cat{C}^\omega \lto \cat{C}^\omega$ defined on objects by $\Psi(X,e) = (\Psi X, \Psi e)$ and a natural isomorphism $\xi: \Psi^\omega \circ \Psi^\omega \lto 1_{\cat{C}^\omega}$ defined by $\xi_{(X,e)} = e \xi_X$. The tuple $(\cat{C}^\omega, \Psi^\omega, \xi^\omega)$ is a supercategory.
\end{remark}

For background on bicategories see \cite{bor94} and the references in \cite{lgdual}. We follow the notation of \cite{lgdual}, so that lower case letters $a,b, \ldots$ denote objects of a bicategory, while upper case letters $X,Y, \ldots$ and greek letters $\alpha, \beta, \ldots$ respectively denote $1$-morphisms and $2$-morphisms. Units are denoted $\Delta$, associators are $\alpha$, and unitors are $\lambda, \rho$. The composition of $Y,X$ is denoted $YX$ or $Y \circ X$.

\begin{definition} A \emph{superbicategory} is a bicategory $\cat{B}$ together with the data:
\begin{itemize}
\item For each object $a$ a $1$-morphism $\Psi_a: a \lto a$ and a $2$-isomorphism $\xi_a: \Psi^2_a \lto \Delta_a$.
\item For each $1$-morphism $X: a \lto b$ a natural $2$-isomorphism $\gamma_X: X \Psi_a \lto \Psi_b X$.
\end{itemize}
This data is required to satisfy the following axioms:
\begin{itemize}
\item For each composable pair $X,Y$ of $1$-morphisms the diagram
\[
\xymatrix@C+2pc@R+1pc{
(YX) \Psi \ar[rrr]^-{\gamma_{YX}} \ar[d]_-{\alpha} & & & \Psi ( YX )\\
Y ( X \Psi ) \ar[r]_-{1_Y * \gamma_X} & Y( \Psi X ) \ar[r]_-{\alpha^{-1}} & ( Y \Psi ) X \ar[r]_-{\gamma_Y * 1_X} & (\Psi Y ) X \ar[u]_-{\alpha}
}
\]
commutes.
\item For every object $a$, $\xi_a * 1_\Psi = 1_\Psi * \xi_a$.
\item For every $1$-morphism $X : a \lto b$, $1_X * \xi_a = ( \xi_b * 1_X ) ( 1_\Psi * \gamma_X ) (\gamma_X * 1_\Psi )$.
\end{itemize}
\end{definition}

\begin{example} Supercategories, superfunctors and supernatural transformations form a superbicategory $\Cat^{\operatorname{sup}}_k$.
\end{example}

A superbicategory can be constructed out of a bicategory $\cat{B}$ in which the categories $\cat{B}(a,b)$ are all equipped with the structure of supercategories; see Appendix \ref{section:constructing_superbicategories}.

\begin{example}\label{example:bicategory_m} There is a bicategory of $\mathbb{Z}_2$-graded $k$-algebras where $1$-morphisms are $\mathbb{Z}_2$-graded bimodules and $2$-morphisms are degree zero bimodule maps. Given a $B$-$A$-module $M$ the shift $M[1]$ has the grading $M[1]_i = M_{i+1}$ and the left and right action given by
\[
b \cdot m = (-1)^{|b|} bm, \qquad m \cdot a = (-1)^{|a|} m a\,.
\]
This is a functor $\Psi = (-)[1]$ on the category of $B$-$A$-bimodules and with $\xi = 1$ this defines the structure of a supercategory on this category of bimodules. The usual isomorphisms of bimodules $\tau$
\begin{align}
N[1] \otimes M &\lto (N \otimes M)[1], \qquad n \otimes m \mapsto n \otimes m \label{eq:shift_iso_a}\\
N \otimes M[1] &\lto (N \otimes M)[1], \qquad n \otimes m \mapsto (-1)^{|n|} n \otimes m\label{eq:shift_iso_b}
\end{align}
satisfy the conditions in Appendix \ref{section:constructing_superbicategories} and therefore give the bicategory of $\mathbb{Z}_2$-graded algebras and bimodules the structure of a superbicategory.
\end{example}

\begin{definition} Given two superbicategories $\cat{B}, \cat{C}$ a \emph{lax superfunctor} $\Phi: \cat{B} \lto \cat{C}$ is a lax functor together with, for each object $a$ in $\cat{B}$, a $2$-morphism
\[
\kappa_a: \Psi_{\Phi a} \lto \Phi( \Psi_a )
\]
satisfying two coherence conditions:
\begin{itemize}
\item[(a)] for all $a$, commutativity of
\[
\xymatrix@C+2pc{
\Psi_{\Phi a} \Psi_{\Phi a} \ar[dd]_-{\xi}\ar[r]^-{\kappa * \kappa} & \Phi( \Psi_a ) \Phi( \Psi_a ) \ar[d]\\
& \Phi( \Psi_a \Psi_a ) \ar[d]^-{\Phi(\xi)}\\
\Delta_{\Phi a} \ar[r] & \Phi( \Delta_a )
}
\]
\item[(b)] for each $1$-morphism $X: a \lto b$ in $\cat{B}$, commutativity of
\[
\xymatrix@C+2pc{
\Phi(X) \Psi_{\Phi a} \ar[d]_-{\gamma} \ar[r]^-{1 * \kappa} & \Phi(X) \Phi(\Psi_a) \ar[r] & \Phi( X \Psi_a ) \ar[d]^-{\Phi(\gamma)}\\
\Psi_{\Phi b} \Phi(X) \ar[r]_-{\kappa * 1} & \Phi(\Psi_b) \Phi(X) \ar[r] & \Phi( \Psi_b X)
}
\]
\end{itemize}
A \emph{strong superfunctor} is a lax superfunctor with $\kappa_a$ an isomorphism for every object $a$.
\end{definition}

\subsection{The superbicategory of Landau-Ginzburg models}\label{section:superbicatLG}

A polynomial $W \in k[x_1,\ldots,x_n]$ is a \emph{potential} if it satisfies the three conditions set out in \cite[Section 2.2]{lgdual}, the two most important being that the partial derivatives $\partial_{x_1} W, \ldots, \partial_{x_n} W$ form a quasi-regular sequence in $k[x]$ and that the quotient $J_W = k[x]/(\partial_{x_1} W, \ldots, \partial_{x_n} W)$ is a finitely generated free $k$-module. Typical examples are the ADE singularities \cite[I \S 2.4]{greuel} of which the simplest are the $A_N$-singularities $W_{A_N} = x_1^{N+1} + x_2^2 + \cdots + x_n^2$ .

A \emph{matrix factorisation} of $W$ over $k[x]$ is a $\mathbb{Z}_2$-graded free $k[x]$-module $X = X^0 \oplus X^1$ together with an odd operator (the differential) $d_X: X \lto X$ satisfying $d_X^2 = W \cdot 1_X$. A matrix factorisation $(X,d_X)$ is \emph{finite rank} if the underlying free module is finitely generated. In this case we may, after choosing a homogeneous basis, write
\[
d_X = \begin{pmatrix} 0 & d_X^1 \\ d_X^0 & 0 \end{pmatrix}
\]
for matrices of polynomials $d_X^0, d_X^1$. Morphisms of matrix factorisations are degree zero $k[x]$-linear maps which commute with the differentials. There is a homotopy relation on morphisms, and the \emph{homotopy category} $\HMF(k[x],W)$ of matrix factorisations has as objects matrix factorisations and as morphisms the equivalence classes of morphisms of matrix factorisations up to homotopy. By $\hmf(k[x],W)$ we denote the full subcategory of finite rank matrix factorisations. For background we refer to \cite{yoshino98}.

By $\hmf(k[x],W)^{\oplus}$ we denote the full subcategory of $\HMF(k[x],W)$ consisting of matrix factorisations which are direct summands (in the homotopy category) of finite rank matrix factorisations. Since $\HMF(k[x],W)$ is idempotent complete, there is an equivalence (see Remark \ref{remark:idempotent_completion} for the notation)
\[
\hmf(k[x],W)^\omega \lto \hmf(k[x],W)^{\oplus}
\]
sending a pair $(X,e)$ to an infinite rank matrix factorisation splitting the idempotent $e$.

If $(X,d_X)$ is a matrix factorisation then so is $(X[1], -d_X)$, and this defines a functor $\Psi = (-)[1]$ on the homotopy category of matrix factorisations. Together with the identity $\xi: \Psi^2 \lto 1$ this makes $\HMF(k[x],W)$ and $\hmf(k[x],W)$ into supercategories.

The bicategory of Landau-Ginzburg models $\LG$ has for its objects pairs $(x,W)$ consisting of an ordered set of variables $x = (x_1,\ldots,x_n)$ and a potential $W \in k[x] = k[x_1,\ldots,x_n]$. Given potentials $W \in k[x]$ and $V \in k[z]$ the supercategory $\cat{LG}(W,V)$ is
\[
\cat{LG}(W,V) = \hmf(k[x,z], V - W)^{\oplus}\,.
\]
Given a third potential $U \in k[y]$ the composition law is a functor
\[
\LG(W,V) \otimes_k \LG(V, U) \lto \LG(W, U)
\]
which sends a pair of $1$-morphisms $X: W \lto V$ and $Y: V \lto U$ to the tensor product
\begin{equation}\label{eq:tensor_comp}
Y \otimes X = ( Y \otimes_{k[z]} X, d_{Y \otimes X} = d_Y \otimes 1 + 1 \otimes d_X )\,.
\end{equation}
This statement requires some care: $k[x,y,z]$ is an infinite rank free module over $k[x,z]$ and so $Y \otimes X$ is an infinite rank matrix factorisation of $U - W$ over $k[x,z]$. However one can prove that $Y \otimes X$ is a direct summand of a finite rank matrix factorisation \cite{dm1102.2957} and therefore a valid object in $\LG(W,U)$.

Let $W \in k[x_1,\ldots,x_n]$ be a potential and $W(x')$ the same polynomial but in a second set of variables $x_1',\ldots,x_n'$. Using formal symbols $\Theta_i$ of odd degree, we define the $k[x,x']$-module
\[
\Delta_W = \bigwedge \big( \bigoplus_{i=1}^n k[x,x'] \cdot \Theta_i \big)\,.
\]
Equipped with a certain differential (see \cite{lgdual}) this is a matrix factorisation of $W(x) - W(x')$ and defines the unit in $\cat{LG}(W,W)$ for composition of $1$-morphisms. For the associator and unitor maps $\rho: X \otimes \Delta_W \lto X$ and $\lambda: \Delta_V \otimes X \lto X$ we also refer to \emph{ibid.}

\begin{lemma} $\LG$ is a superbicategory.
\end{lemma}
\begin{proof}
The categories $\LG(W,V)$ are all naturally supercategories, and the same isomorphisms as in \eqref{eq:shift_iso_a}, \eqref{eq:shift_iso_b} define the necessary natural isomorphisms $\tau$ to equip $\LG$ with the structure of a superbicategory with $\Psi_W = \Delta_W[1]$, using Appendix \ref{section:constructing_superbicategories}.
\end{proof}

The superbicategory structure on $\LG$ is used implicitly in \cite[Section 7]{lgdual}.

\subsection{Clifford algebras}\label{section:clifford_algs}

In a cut system the composition rule produces objects which are representations of Clifford algebras. We briefly recall the basic theory of these algebras and their modules; for a full discussion see \cite{friedrich}. For $n \ge 0$ the Clifford algebra $A_n$ is the associative $\mathbb{Z}_2$-graded $k$-algebra generated by odd elements $a_1,\ldots,a_n$ and $a_1^\dagger, \ldots, a_n^\dagger$ subject to the following relations
\[
[a_i, a_j] = 0, \qquad [a_i^\dagger, a_j^\dagger] = 0, \qquad [a_i, a_j^\dagger] = \delta_{ij}\,.
\]
for all $1 \le i, j \le n$. In this paper all our commutators are graded, $[a,b] = ab - (-1)^{|a||b|} ba$. We set $A_0 = k$. The algebras $A_n$ are Morita trivial in the sense that they are isomorphic to the endomorphism algebras of finite rank free $\mathbb{Z}_2$-graded $k$-modules. 

Indeed, if $F_n = k \theta_1 \oplus \cdots \oplus k \theta_n$ with $|\theta_i| = 1$ for all $i$, then

\begin{lemma} With $S_n = \bigwedge F_n$ the map $A_n \lto \End_k(S_n)$ defined by
\[
a_i \mapsto \theta_i \wedge (-), \qquad a_i^\dagger \mapsto \theta_i^*\, \lrcorner\, (-)
\]
is an isomorphism of $\mathbb{Z}_2$-graded $k$-algebras, where $\lrcorner$ denotes contraction.
\end{lemma}

In particular this means that every $\mathbb{Z}_2$-graded $A_n$-module is isomorphic to $S_n \otimes_k V$ for some $\mathbb{Z}_2$-graded $k$-module $V$. From the isomorphism of $k$-modules
\[
S_m \otimes_k S_n = \bigwedge F_m \otimes_k \bigwedge F_n \cong \bigwedge F_{m+n} = S_{m+n}
\]
we deduce an isomorphism of algebras
\begin{equation}\label{eq:algebra_A_additive}
A_m \otimes_k A_n \cong A_{m+n}\,.
\end{equation}
Given $m, n \ge 0$ we introduce notation for the $A_m$-$A_n$-bimodule
\begin{equation}
S_{m,n} = S_m \otimes_k S_n^*
\end{equation}
where $S_n^* = \Hom_k(S_n, k)$. There is an isomorphism of $A_l$-$A_n$-bimodules
\begin{equation}\label{eq:isosbimodule}
S_{l,m} \otimes_{A_m} S_{m,n} = S_l \otimes_k S_m^* \otimes_{A_m} S_m \otimes_k S_n^* \cong S_{l,n}\,.
\end{equation}
In the case $k = \mathbb{C}$, $S_n$ is often referred to as the \emph{spinor representation} \cite[p.14]{friedrich}.

\begin{remark} Recall that given a finite rank free $k$-module $V$ and a symmetric bilinear form $B: V \otimes_k V \lto k$ the associated Clifford algebra $\operatorname{Cl}(V, B)$ is the associative $k$-algebra generated by the elements of $V$ subject to the relations
\[
vw + wv = B(v,w) \cdot 1\,.
\]
This is a $\mathbb{Z}_2$-graded $k$-algebra when we assign $|v| = 1$ for all $v \in V$ and $|1| = 0$. If we take $V = F \oplus F^*$ with the bilinear form $B$ under which $B(F,F) = 0, B(F^*, F^*) = 0$ and for $x \in F$ and $\nu \in F^*$, $B(x, \nu) = \frac{1}{2} \nu(x)$, then $A_n$ is the Clifford algebra of the pair $(V,B)$.
\end{remark}

\subsection{The Clifford thickening}

Let $\cat{C}$ be a small idempotent complete supercategory. We construct a new supercategory $\cat{C}^\bullet$ called the \emph{Clifford thickening} in which the objects are pairs $(X,n)$ of an integer $n \ge 0$ and a left $A_n$-module $X$ in $\cat{C}$.

We will need the operation of tensoring objects of $\cat{C}$ with $k$-modules. This is standard material, which is recalled in Appendix \ref{section:tensorproduct_supcat}. The upshot is that if $A$ is a Morita trivial $\mathbb{Z}_2$-graded $k$-algebra and if $X \in \cat{C}$ is a left $A$-module, while $V$ is a $\mathbb{Z}_2$-graded right $A$-module (in the usual sense, i.e. in the category of $k$-modules) then there is an object $V \otimes_A X$ in $\cat{C}$ which is functorial in both $V$ and $X$. Morever if $V$ is a $B$-$A$-bimodule then $V \otimes_A X$ is naturally a left $B$-module, and so on. If $A$ is a $\mathbb{Z}_2$-graded $k$-algebra then $\cat{C}_A$ denotes the supercategory of $A$-modules in $\cat{C}$ with $A$-linear maps.

Similarly to Example \ref{example:bicategory_m} there is a superbicategory $\cat{M}$ of Morita trivial $\mathbb{Z}_2$-graded $k$-algebras: $1$-morphisms are $\mathbb{Z}_2$-graded bimodules which are finitely generated and projective over $k$ and $2$-morphisms are degree zero bimodule maps.

\begin{proposition} The assignment of the supercategory $\cat{C}_A$ to an algebra $A$ and of the superfunctor $\Phi_V = V \otimes_A (-)$ to a $B$-$A$-bimodule $V$ determines a strong superfunctor
\[
\Phi^{\cat{C}}: \cat{M} \lto \Cat^{\operatorname{sup}}_k
\]
to the superbicategory of small supercategories and superfunctors.
\end{proposition}
\begin{proof}
The isomorphism $\Phi_W \circ \Phi_V \cong \Phi_{W \otimes V}$ is given by Lemma \ref{lemma:assoctensor}, and apart from that the only checking that needs to be done are straightforward coherence diagrams.
\end{proof}

Next we construct a strong functor into $\cat{M}$ which picks out the Clifford algebras.

\begin{definition}
Let $\mathbb{N}$ denote the category of integers $n \ge 0$ with a unique morphism $\phi_{m,n}: n \lto m$ for each pair $m,n$. %This category is monoidal with $m \otimes n = m + n$.
\end{definition}

We view $\mathbb{N}$ as a bicategory with only identity $2$-morphisms.

\begin{lemma} There is a strong functor $\mathbb{N} \lto \cat{M}$ defined by
\begin{gather*}
n \mapsto A_n = \End_k( S_n ),\\
\phi_{m,n} \mapsto S_{m,n} = S_m \otimes_k S_n^*\,.
\end{gather*}
\end{lemma}

The composite of these strong functors is a strong functor
\begin{equation}\label{eq:strongfunctorgroth}
\xymatrix@C+1pc{
\mathbb{N} \ar[r] & \cat{M} \ar[r]^-{\Phi^{\cat{C}}} & \Cat^{\operatorname{sup}}_k
}
\end{equation}
sending $n$ to the category of left $A_n$-modules in $\cat{C}$ and $\phi_{m,n}$ to the functor $S_{m,n} \otimes_{A_n} -$. There is a close connection between strong functors into the bicategory of small categories, and what are called cofibered categories, given by the Grothendieck construction \cite{vistoli}.

\begin{definition} The \emph{Clifford thickening} $\cat{C}^\bullet$ of the supercategory $\cat{C}$ is the category which results from the Grothendieck construction applied to the strong functor \eqref{eq:strongfunctorgroth}.
\end{definition}

More concretely $\cat{C}^\bullet$ is the category with objects all pairs $(X,n)$ of an integer $n \ge 0$ and an object $X \in \cat{C}$ with a left $A_n$-action. A morphism $\alpha: (X,n) \lto (Y,m)$ is a morphism of $A_m$-modules $\alpha: S_{m,n} \otimes_{A_n} X \lto Y$. Composition of a pair of morphisms
\begin{equation}\label{eq:comp_chain0}
\xymatrix{
(X,n) \ar[r]^-{\alpha} & (Y,m) \ar[r]^-{\beta} & (Z,l)
}
\end{equation}
is given by the morphism of $A_l$-modules
\begin{equation}\label{eq:comp_chain1}
\xymatrix@C+1pc{
S_{l, n} \otimes_{A_n} X \ar[r]^-{\cong} & (S_{l,m} \otimes_{A_m} S_{m,n}) \otimes_{A_n} X \ar[d]^-{\cong}\\
& S_{l,m} \otimes_{A_m} ( S_{m,n} \otimes_{A_n} X ) \ar[r]^-{ 1 \otimes \alpha } & S_{l,m} \otimes_{A_m} Y \ar[r]^-{\beta} & Z\,.
}
\end{equation}
There is a canonical functor $p: \cat{C}^\bullet \lto \mathbb{N}$ defined by $p(X,n) = n$, which defines a cofibered category over $\mathbb{N}$ by the definition of the Grothendieck construction.

\begin{lemma} $\cat{C}^\bullet$ is a supercategory and the functor $\iota: \cat{C} \lto \cat{C}^\bullet$ defined by $\iota(X) = (X,0)$ is an equivalence of supercategories.
\end{lemma}
\begin{proof}
The supercategory structure is given by the functor $\Psi: \cat{C}^\bullet \lto \cat{C}^\bullet$ where $\Psi(X,n) = ( \Psi X, n )$ and $\Psi X$ has the $A_n$-action given by Definition \ref{defn:psimodule}. 

The functor $\iota$ is fully faithful. To see that it is essentially surjective, let $(X,n) \in \cat{C}^\bullet$ be given. We show that $(X,n)$ is isomorphic to $(\widetilde{X},0)$ where $\widetilde{X} = S_n^* \otimes_{A_n} X$. This follows from the fact that the isomorphism
\[
S_n \otimes_k \widetilde{X} = S_n \otimes_k S_n^* \otimes_{A_n} X \cong X
\]
defines an isomorphism $(\widetilde{X}, 0) \lto (X,n)$ in $\cat{C}^\bullet$.
\end{proof}

\begin{definition}\label{defn:idempotent_e} For $n \ge 0$ let $e_n \in A_n$ denote the element
\[
e_n = a_1^\dagger \cdots a_n^\dagger a_n \cdots a_1\,.
\]
This is the idempotent corresponding to the summand $k \cdot 1$ of $S_n$.
\end{definition}

\begin{remark}\label{remark:tensor_and_idempotent} It may seem odd to labour to construct a category $\cat{C}^\bullet$ equivalent to $\cat{C}$. The point is that while $(X,n)$ and $(\widetilde{X}, 0)$ are isomorphic in $\cat{C}^\bullet$, they still represent quantitatively different states of knowledge. If $\widetilde{X}$ is known then $X$ may be easily constructed by taking direct sums, but if $X$ is known then forming the tensor product $\widetilde{X}$ is equivalent to splitting the idempotent $e_n: X \lto X$ and this requires work.
% If $X$ is an $A_n$-module in $\cat{C}$, then the defining property of $\widetilde{X} = S_n^* \otimes_{A_n} X$ is that there is a morphism $\eta: X \lto \widetilde{X}$ satisfying $\eta \circ a_i = 0$ for all $1 \le i \le n$ and which is universal with this property. That is, given any morphism $\delta:X  \lto Y$ with this property there is a unique morphism $\kappa: \widetilde{X} \lto Y$ such that $\kappa \circ \eta = \delta$. A morphism $\delta: X \lto Y$ satisfies $\delta \circ a_i = 0$ for $1 \le i \le n$ if and only if $\delta \circ ( 1 - e_n ) = 0$.

For each $n$ the fiber of the functor $p: \cat{C}^\bullet \lto \mathbb{N}$ over $n$ is $\cat{C}_{A_n}$. We may depict the cofibered category $p$ as follows:
\begin{equation}\label{eq:cofibered_diagram}
\xymatrix@C+3pc@R+1pc{
\cat{C} \ar@{.}[d] \ar@<1ex>[rrr]^-{E} & & & \cat{C}_{A_n} \ar@{.}[d] \ar@<1ex>[lll]^-{F}\\
0 \ar@{.>}@<0.8ex>[r] & 1 \ar@{.>}@<0.8ex>[r]\ar@{.>}@<0.8ex>[l] & \cdots \ar@{.>}@<0.8ex>[l]\ar@{.>}@<0.8ex>[r] & n \ar@{.>}@<0.8ex>[l]
}\,.
\end{equation}
The transport between the fibers is given by the equivalences
\[
E(-) = S_{n} \otimes_k (-), \qquad F(-) = S^*_{n} \otimes_{A_n} (-)\,.
\]
While these functors are adjoint, and therefore uniquely determine one another, we have explained that there is a fundamental asymmetry between them from the point of view of computation: knowledge of an object of $\cat{C}^\bullet$ over $n$ represents a state of higher uncertainty than knowledge of an object over $m < n$, given that in the end both objects ``only'' contain the information of the corresponding object $\widetilde{X}$ over zero, but more work is necessary to extract this information from an object of $\cat{C}_{A_n}$ than from an object of $\cat{C}_{A_m}$.
\end{remark}

\begin{remark}\label{remark:tensor_in_steps} Given $(X,n) \in \cat{C}^\bullet$, one way to compute the tensor product
\[
S_{n+1,n} \otimes_{A_n} X = (S_{n+1} \otimes_k S_n^*) \otimes_{A_n} X \cong S_{n+1} \otimes_k ( S_n^* \otimes_{A_n} X )
\]
is to compute $S_n^* \otimes_{A_n} X$ and then take direct sums. Alternatively,
\begin{align*}
S_{n+1,n} \otimes_{A_n} X &= S_{n+1} \otimes_k S_n^* \otimes_{A_n} X\\
&\cong S_1 \otimes_k S_n \otimes_k S_n^* \otimes_{A_n} X \\
&\cong S_1 \otimes_k A_n \otimes_{A_n} X \\
&\cong S_1 \otimes_k X\,.
\end{align*}
That is, if we equip $S_1 \otimes_k X = X \oplus \Psi X$ with an $A_{n+1} \cong A_1 \otimes A_n$ action by making $A_n$ act on $X$ and $A_1$ act on $S_1$, this computes the tensor product. In a similar way, $S_{n-1,n} \otimes_{A_n} X$ is computed by splitting the idempotent $a_1^\dagger a_1$ on $X$, so that the functor $F$ of \eqref{eq:cofibered_diagram} may be computed in $n$ steps by splitting $a_1^\dagger a_1$, then $a_2^\dagger a_2$, and so on.
\end{remark}
%TODO note: $S_{m,n} \otimes_{A_n}$ is an equivalence so all fibers are isomorphic to $\cat{C}$.

\begin{remark} An alternative description of morphisms in $\cat{C}^\bullet$ follows from Lemma \ref{lemma:morphisms_two_forms}. This result tells us that given an $A_n$-module $X$ and $A_m$-module $Y$ in $\cat{C}$ there is a bijection between morphisms of $A_m$-modules $\alpha: S_{m,n} \otimes_{A_n} X \lto Y$ and morphisms
\[
\alpha_0: X \lto Y \text{ satisfying } \alpha_0 \circ a_i = 0 \text{ for } 1 \le i \le n \text{ and } a_i^\dagger \circ \alpha_0 = 0 \text{ for } 1 \le i \le m\,.
\]
The bijection is defined by sending $\alpha$ to the composite ($\iota$ is defined by $\iota(1) = 1 \otimes 1^*$)
\[
\xymatrix@C+2pc{
X \cong A_n \otimes_{A_n} X \ar[r]^-{\iota \otimes 1} & S_{m,n} \otimes_{A_n} X \ar[r]^-{\alpha} & Y
}\,
\]
Given a chain of composable morphisms as in \eqref{eq:comp_chain0} there are corresponding morphisms $\alpha_0: X \lto Y$ and $\beta_0: Y \lto Z$ and it can be checked that the morphism corresponding under the bijection of Lemma \ref{lemma:morphisms_two_forms} to the composite $\beta \circ \alpha$ given in \eqref{eq:comp_chain1} is
\[
(\beta \circ \alpha)_0 = \beta_0 \circ \alpha_0\,.
\]
\end{remark}

\section{Cut systems}\label{section:cut_systems}

We begin with the definition of cut functors.

\begin{definition} Given supercategories $\cat{C}_1,\cat{C}_2,\cat{C}_3$ a \emph{superfunctor} $T: \cat{C}_1 \otimes_k \cat{C}_2 \lto \cat{C}_3$ is a functor together with natural isomorphisms
\begin{align*}
\tau &: T \circ ( \Psi \otimes 1 ) \lto \Psi \circ T,\\
\tau &: T \circ ( 1 \otimes \Psi ) \lto \Psi \circ T
\end{align*}
making the following diagrams commute:
\begin{equation}\label{eq:psicomp1}
\xymatrix@R+1pc@C+1pc{
T( 1 \otimes \Psi^2 ) \ar[r]^-{\tau}\ar[d]_-{T(1 \otimes \xi)} & \Psi T( 1 \otimes \Psi ) \ar[r]^-{\tau} & \Psi^2 T \ar[d]^-{\xi * 1_T}\\
T \ar[rr]_-{1_T} & & T
}
\end{equation}
\begin{equation}\label{eq:psicomp2}
\xymatrix@R+1pc@C+1pc{
T( \Psi^2 \otimes 1 ) \ar[r]^-{\tau} \ar[d]_-{T (\psi \otimes 1)}  & \Psi T( \Psi \otimes 1 ) \ar[r]^-{\tau} & \Psi^2 T \ar[d]^-{\xi * 1_T}\\
T \ar[rr]_-{1_T} & & T
}
\end{equation}
\begin{equation}\label{eq:psicomp3}
\xymatrix@R+1pc@C+1pc{
T( \Psi \otimes \Psi ) \ar[r]^-{\tau}\ar[d]_-{-\tau} & \Psi T( 1 \otimes \Psi ) \ar[d]^-{\tau}\\
\Psi T( \Psi \otimes 1 ) \ar[r]_-{\tau} & \Psi^2 T
}
\end{equation}

\end{definition}

\begin{definition}\label{defn:cutfunctor1} A \emph{cut functor of degree $t$} on a triple $\cat{C}_1, \cat{C}_2,\cat{C}_3$ of idempotent complete small supercategories is a superfunctor $T: \cat{C}_1 \otimes_k \cat{C}_2 \lto \cat{C}_3^\bullet$ such that
\[
\xymatrix{
\cat{C}_1 \times \cat{C}_2 \ar[r]^-{T} & \cat{C}_3^\bullet \ar[r]^-{p} & \mathbb{N}
}
\]
sends every object to the integer $t \in \mathbb{N}$.
\end{definition}

With the above notation:

\begin{proposition}\label{prop:extend_cut_functor} Given a cut functor $T$ of degree $t$ on $\cat{C}_1, \cat{C}_2,\cat{C}_3$ there is a canonical superfunctor $\widetilde{T}: \cat{C}_1^\bullet \otimes_k \cat{C}_2^\bullet \lto \cat{C}_3^\bullet$ with the property that the diagram
\[
\xymatrix{
\cat{C}_1 \times \cat{C}_2 \ar[dr]_-{\iota \times \iota} \ar[rr]^-{T} & & \cat{C}_3^\bullet\\
& \cat{C}_1^\bullet \times \cat{C}_2^\bullet \ar[ur]_-{\widetilde{T}}
}
\]
commutes, and for every pair of objects $X \in \cat{C}_1^\bullet$ and $Y \in \cat{C}_2^\bullet$
\[
p \widetilde{T}( X, Y ) = p( X ) + p( Y ) + t\,.
\]
We refer to $\widetilde{T}$ as the \emph{extension} of $T$.
%and
%\[
%\xymatrix@C+2pc{
%\cat{C}_1^\bullet \times \cat{C}_2^\bullet \ar[d]_-{p \times p}\ar[rr]^-{\widetilde{T}} & & \cat{C}_3^\bullet \ar[dd]^-{p}\\
%\mathbb{N} \times \mathbb{N} \ar[d]_-{\cong}\\
%\mathbb{N} \times \{ * \} \times \mathbb{N} \ar[r]_-{1 \times t \times 1} & \mathbb{N} \times \mathbb{N} \times \mathbb{N} \ar[r]_-{\otimes} & \mathbb{N}
%}
%\]
%commute. 
\end{proposition}
\begin{proof}
The proof is given in Appendix \ref{app:proof_cut_extension}.
\end{proof}

We will use \emph{effectively computable} as a synonym for the precise notion of a \emph{computable} function \cite{enderton}, since ``computable'' often has a more colloquial meaning in our context.

% This is the definition from cutsys10
\begin{definition} A \emph{cut system} $\cat{A}$ is determined by the following data:
\begin{itemize}
\item A class $|\cat{A}|$ of objects and for each object $a$ an integer $\pi(a) \ge 0$.
\item For each pair $a,b$ of objects a small idempotent complete supercategory $\cat{A}(a,b)$. 
\item For each triple $a,b,c$ of objects a cut functor of degree $\pi(b)$
\[
T: \cat{A}(a,b) \otimes_k \cat{A}(b,c) \lto \cat{A}(a,c)^{\bullet}\,.
\]
The image of a pair $X \otimes Y$ under the canonical extension
\[
\widetilde{T}: \cat{A}(a,b)^\bullet \otimes_k \cat{A}(b,c)^\bullet \lto \cat{A}(a,c)^\bullet
\]
is denoted $Y \l X$ and is called the \emph{cut of $Y$ and $X$ along $b$}.
\item For each object $a$, an object $\Delta_a \in \cat{A}(a,a)$.
\item For objects $a,b,c,d$ a natural isomorphism $\alpha$ as shown below:
\[
\xymatrix@C+2pc{
\cat{A}(a,b) \times \cat{A}(b,c) \times \cat{A}(c,d) \ar[d]_-{T \times 1} \ar[r]^-{1 \times T} & \cat{A}(a,b) \times \cat{A}(b,d)^\bullet \ar[d]^{\widetilde{T}} \ar@{=>}[dl]^-{\alpha}\\
\cat{A}(a,c)^\bullet \times \cat{A}(c,d) \ar[r]_-{\widetilde{T}} & \cat{A}(a,d)^\bullet
}
\]
That is, for $X \in \cat{A}(a,b), Y \in \cat{A}(b,c), Z \in \cat{A}(c,d)$ an isomorphism
\[
\alpha_{Z,Y,X}: ( Z \l Y ) \l X \lto Z \l ( Y \l X )
\]
of $A_{\pi(b)} \otimes_k A_{\pi(c)}$-modules in $\cat{A}(a,d)$, natural in all three variables.
\item For objects $a,b$ natural isomorphisms $\mu, \eta$ as shown below:
\[
\xymatrix@C+2pc{
1 \times \cat{A}(a,b) \ar[d]_-{\iota_{\Delta} \times 1} & \cat{A}(a,b) \ar@{=>}@<4ex>[l]_-{\eta} \ar@{=>}@<-4ex>[r]^-{\mu} \ar[d]^-{\iota} \ar[l]_-{\cong} \ar[r]^-{\cong} & \cat{A}(a,b) \times 1 \ar[d]^-{1 \times \iota_{\Delta}}\\
\cat{A}(a,a) \times \cat{A}(a,b) \ar[r]_-{T} & \cat{A}(A,B)^\bullet & \cat{A}(a,b) \times \cat{A}(b,b) \ar[l]^-{T}
}
\]
where $1$ is the category with one object and one morphism and $\iota_{\Delta}$ is the functor sending this object to $\Delta$. That is, for each $X \in \cat{A}(a,b)$ a pair of isomorphisms
\begin{equation}\label{eq:defn_cutsys_lambda}
\mu: X \lto \Delta_b \l X, \quad \eta: X \lto X \l \Delta_a
\end{equation}
in $\cat{A}(a,b)^\bullet$ natural in $X$.
\end{itemize}
This data is subject to the following coherence conditions:
\begin{itemize}
\item[(1)] For objects $a,b,c,d,e$ and $X \in \cat{A}(a,b),Y \in \cat{A}(b,c),Z \in \cat{A}(c,d),Q \in \cat{A}(d,e)$, commutativity of the diagram
\begin{equation}
\xymatrix@C+2pc{
((Q \l Z) \l Y ) \l X \ar[d]_-{\alpha_{Q \l Z, Y, X}}\ar[r]^-{\alpha_{Q,Z,Y} \l 1} & (Q \l (Z \l Y)) \l X \ar[r]^-{\alpha_{Q, Z \l Y, X}} & Q \l ( ( Z \l Y ) \l X ) \ar[d]^-{1 \l \alpha_{Z,Y,X}}\\
(Q \l Z) \l (Y \l X) \ar[rr]_-{\alpha_{Q, Z, Y \l X}} & & Q \l ( Z \l ( Y \l X ) )
}
\end{equation}
Note that when we write $\alpha_{Q, Z \l Y, X}$ we refer only the underlying object of $Z \l Y$.
\item[(2)] For objects $a,b$ and $X \in \cat{A}(a,b), Y \in \cat{A}(b,c)$ commutativity of the diagram
\begin{equation}\label{eq:unitor_constraint_cut}
\xymatrix{
(Y \l \Delta_b ) \l X \ar[rr]^-{\alpha} & & Y \l ( \Delta_b \l X )\\
& Y \l X \ar[ul]^-{\eta \l 1_X} \ar[ur]_-{1_Y \l \mu}
}
\end{equation}
\item[(3)] commutativity of \eqref{eq:psiunitor1},\eqref{eq:psiunitor2} for every $X$ and \eqref{eq:psithirdlastdia}, \eqref{eq:psimiddledia}, \eqref{eq:psilastdia} for every composable triple $X,Y,Z$.
\end{itemize}
Finally, the data is subject to three computability conditions:
\begin{itemize}
\item[(1)] For each pair of objects $a,b$ the category $\cat{A}(a,b)$ is effectively computable, in the sense that the morphisms and objects are described by finite data and the composition rule is computable.
\item[(2)] For each triple $a,b,c$ of objects the associated cut functor is effectively computable. In particular, for $1$-morphisms $Y$ and $X$ the object $Y \l X$ and its Clifford action are effectively computable from $X$ and $Y$.
\item[(3)] The associators and unitors are effectively computable from their arguments.
\end{itemize}
\end{definition}

\begin{remark} To give natural isomorphisms \eqref{eq:defn_cutsys_lambda} in $\cat{A}(a,b)^\bullet$ is equivalent to giving natural isomorphisms in $\cat{A}(a,b)$
\begin{equation}\label{eq:unitor_tilde_pre}
\mu: S_{\pi(b)} \otimes_k X \lto \Delta_b \l X, \quad \eta: S_{\pi(a)} \otimes_k X \lto X \l \Delta_a
\end{equation}
which are respectively $A_{\pi(b)}$ and $A_{\pi(a)}$-linear. 

By Lemma \ref{lemma:morphism_out_1} such morphisms correspond bijectively to morphisms in $\cat{A}(a,b)$
\begin{equation}\label{eq:unitor_tilde}
\mu_0: X \lto \Delta_b \l X, \quad \eta_0: X \lto X \l \Delta_a
\end{equation}
satisfying respectively $a_i^\dagger \circ \mu_0 = 0$ for $1 \le i \le m$ and $a_i^\dagger \circ \eta_0 = 0$ for $1 \le i \le n$. The reason to introduce these maps is that commutativity of \eqref{eq:unitor_constraint_cut} can be more conveniently expressed in terms of the maps in \eqref{eq:unitor_tilde}, see Lemma \ref{lemma:unitor_check_help} below.
\end{remark}

Let $\cat{A}$ be a cut system. There is a superbicategory $\cat{A}^\bullet$ with objects the class of objects of $\cat{A}$ and with $1$- and $2$-morphisms defined by $\cat{A}^\bullet(a,b) = \cat{A}(a,b)^\bullet$. The units of $\cat{A}^\bullet$ are the objects $\Delta_a$ in the cut system, and the composition in the bicategory is given by the canonical extensions $\widetilde{T}$. To complete the definition of a superbicategory it only remains to give the associators and unitors, and the super-structure.

\begin{lemma} Given objects $a,b,c,d$ the natural isomorphism $\alpha$ canonically extends to a natural isomorphism $\widetilde{\alpha}$ as in the diagram
\[
\xymatrix@C+2pc{
\cat{A}(a,b)^\bullet \times \cat{A}(b,c)^\bullet \times \cat{A}(c,d)^\bullet \ar[d]_-{\widetilde{T} \times 1} \ar[r]^-{1 \times \widetilde{T}} & \cat{A}(a,b)^\bullet \times \cat{A}(b,d)^\bullet \ar[d]^{\widetilde{T}} \ar@{=>}[dl]^-{\widetilde{\alpha}}\\
\cat{A}(a,c)^\bullet \times \cat{A}(c,d)^\bullet \ar[r]_-{\widetilde{T}} & \cat{A}(a,d)^\bullet
}
\]
\end{lemma}
\begin{proof}
Suppose $(X,n) \in \cat{A}(a,b)^\bullet, (Y,m) \in \cat{A}(b,c)^\bullet$ and $(Z,t) \in \cat{A}(c,d)^\bullet$ are given. The underlying object of $((Z,t) \l (Y, m)) \l (X,n)$ is $(Z \l Y) \l X$, and it is clear that $\alpha_{Z,Y,X}$ is $A_t \otimes A_m \otimes A_n$-linear and therefore defines an isomorphism in $\cat{A}(a,d)^\bullet$
\[
\alpha_{Z,Y,X}: ((Z,t) \l (Y, m)) \l (X,n) \lto (Z,t) \l ((Y, m) \l (X,n))\,.
\]
Naturality is straightforward.
\end{proof}

\begin{lemma} Given objects $a,b$ the natural isomorphisms $\eta, \mu$ canonically extend to natural isomorphisms $\widetilde{\eta}, \widetilde{\mu}$ as in the diagram
\[
\xymatrix@C+2pc{
1 \times \cat{A}(a,b)^\bullet \ar[d]_-{\iota_{\Delta} \times 1} & \cat{A}(a,b)^\bullet \ar@{=>}@<4ex>[l]_-{\widetilde{\eta}} \ar@{=>}@<-4ex>[r]^-{\widetilde{\mu}} \ar[d]^-{1} \ar[l]_-{\cong} \ar[r]^-{\cong} & \cat{A}(a,b)^\bullet \times 1 \ar[d]^-{1 \times \iota_{\Delta}}\\
\cat{A}(a,a)^\bullet \times \cat{A}(a,b)^\bullet \ar[r]_-{\widetilde{T}} & \cat{A}(A,B)^\bullet & \cat{A}(a,b)^\bullet \times \cat{A}(b,b)^\bullet \ar[l]^-{\widetilde{T}}
}
\]
 \end{lemma}
\begin{proof}
Let $(X,n) \in \cat{A}(a,b)^\bullet$. Forgetting the $A_n$-action on $X$, we have isomorphisms $\eta$ and $\mu$ as in \eqref{eq:unitor_tilde_pre}, which are respectively $A_{\pi(b)}$ and $A_{\pi(a)}$-linear. Since these isomorphisms are natural, they are automatically $A_n$-linear. Verifying that these isomorphisms are natural with respect to the larger class of morphisms in $\cat{A}(a,b)^\bullet$ is straightforward.
\end{proof}

The supercategory structure is given as part of the cut system, so we have:

% The above definition of a cut system is from cutsys10. This is by p.11 there something which gives rise to a cut system in the sense of cutsys6, which is proved there to define a superbicategory.
\begin{proposition} Given a cut system $\cat{A}$, $\cat{A}^\bullet$ is a superbicategory.
\end{proposition}
\begin{proof}
This follows from Lemma \ref{lemma:constructingsuper}, with associator $\widetilde{\alpha}$ and unitors $\lambda =\widetilde{\mu}^{-1}, \rho = \widetilde{\eta}^{-1}$. The coherence follows from the coherence axioms of the cut system.
\end{proof}

We often confuse a cut system $\cat{A}$ with the associated superbicategory. For example, we refer to objects $X$ of $\cat{A}(a,b)^\bullet$ as a $1$-morphisms. To distinguish objects of the original category $\cat{A}(a,b)$ we refer to these objects as \emph{normal} or \emph{cut-free} $1$-morphisms. We keep the notation $Y \l X$ for the composition rule in $\cat{A}^\bullet$.

The following lemma is useful in verifying the unitor constraint.

\begin{lemma}\label{lemma:unitor_check_help} The commutativity of \eqref{eq:unitor_constraint_cut} is equivalent to commutativity of
\begin{equation}\label{eq:unitor_check_help1}
\xymatrix@C+1pc{
(Y \l \Delta_b) \l X \ar[rr]^-{\alpha} & & Y \l ( \Delta_b \l X )\\
Y \l X \ar[u]^-{\eta_0 \l 1} & Y \l X \ar[l]^-{e_{\pi(b)}} \ar[r]_-{e_{\pi(b)}} & Y \l X \ar[u]_-{1 \l \mu_0}
}
\end{equation}
where $e_{\pi(b)}$ is the idempotent of Definition \ref{defn:idempotent_e}.
\end{lemma}
\begin{proof}
Let $m = \pi(b)$. Then $(Y \l \Delta_b) \l X$ has two $A_m$ actions, one from $Y \l \Delta_b$ which we denote by $A_{m_1}$, and the other from the cut with $X$ which we denote by $A_{m_2}$. By definition of the canonical extension $\widetilde{T}$ in Appendix \ref{app:proof_cut_extension}, $\eta \l 1$ is the $A_{m_1+m_2}$-linear morphism
\[
\eta: S_{m_1 + m_2, m_2} \otimes_{A_{m_2}} ( Y \l X ) \lto (Y \l \Delta_b ) \l X
\]
which corresponds under Lemma \ref{lemma:morphisms_two_forms} to
\[
\xymatrix@C+1pc{
Y \l X \ar[r]^-{e_m} & Y \l X \ar[r]^-{\eta_0 \l 1} & (Y \l \Delta_b) \l X
}\,.
\]
Applying the same logic to $\mu$ completes the proof.
\end{proof}

In a cut system the unitors $\mu, \eta$ go in the opposite direction to what one would expect from the conventions of bicategories. This is because it is more natural to describe morphisms out of a cut-free $1$-morphism and into a $1$-morphism carrying a nontrivial Clifford representation, than in the other direction.

\subsection{Cut models}\label{section:cut_models}

\begin{definition} A \emph{cut model} for a superbicategory $\cat{B}$ is a cut system $\cat{A}$ with the same underlying class of objects as $\cat{B}$ and a strong superfunctor $I: \cat{B} \lto \cat{A}^\bullet$ with the property that for every pair of objects $a,b$ the functor
\[
I_{a,b}: \cat{B}(a,b) \lto \cat{A}(a,b)^\bullet
\]
factors as
\[
\xymatrix{
\cat{B}(a,b) \ar[rr]^-{I_{a,b}} \ar[dr]_-{J_{a,b}} & & \cat{A}(a,b)^\bullet\\
& \cat{A}(a,b) \ar[ur]_-{\iota}
}
\]
where $J_{a,b}$ is an equivalence of supercategories. We often identify $X$ with $I(X)$.
\end{definition}

A cut model is an effectively computable model for composition of $1$-morphisms in~$\cat{B}$. Given $X: a \lto b$ and $Y: b \lto c$, let us suppose that the composition $Y \circ X$ in $\cat{B}$ is not defined algorithmically: perhaps this object is only identified up to unique isomorphism by a universal property. In contrast, the cut $Y \l X$ is computable from $X$ and $Y$, and the strong superfunctor $I$ provides an isomorphism in $\cat{A}(a,c)^\bullet$
\begin{equation}
Y \l X \cong Y \circ X\,.
\end{equation}
Here $Y \circ X$ is embedded into $\cat{A}^\bullet$ as a cut-free $1$-morphism and the cut $Y \l X$ by definition comes with an $A_{\pi(b)}$-module structure. That is, there is an isomorphism of $A_{\pi(b)}$-modules
\begin{equation}\label{eq:cutvsoriginal}
Y \l X \cong S_{\pi(b)} \otimes_k (Y \circ X )
\end{equation}
or what is the same, an isomorphism of objects in $\cat{A}(a,b)$
\begin{equation}\label{eq:cutvsoriginal2}
S^*_{\pi(b)} \otimes_{A_{\pi(b)}} ( Y \l X ) \cong Y \circ X \,.
\end{equation}
The general philosophy is that a cut model decomposes composition in $\cat{B}$ into two stages: 
\begin{itemize}
\item first form the cut $Y \l X$, and then
\item tensor $Y \l X$ with $S^*_{\pi(b)}$ to compute $Y \circ X$. 
\end{itemize}
The idea is that the desired composition $Y \circ X$ lies inside $Y \l X$ in a way that is ``known'' in the sense that it is encoded by the Clifford action, but which is only \emph{realised} or \emph{constructed} after a computation is done. Forming the tensor product $S^*_{\pi(b)} \otimes_{A_{\pi(b)}} ( Y \l X )$ starting with the concretely known object $Y \l X$ can be further decomposed into a $\pi(b)$-step process of ``atomic'' computations, using the diagram of Remark \ref{remark:tensor_and_idempotent}
\begin{equation}\label{eq:computation_fiber_diagram}
\xymatrix@C+1pc@R+1pc{
Z_0 \cong Y \circ X \ar@{.}[d] & Z_1 \ar@{.}[d] & & Z_{\pi(b)-1} \ar@{.}[d] & Z_{\pi(b)} = Y \l X \ar@{.}[d]\\
0 \ar@{.>}@<0.8ex>[r] & 1 \ar@{.>}@<0.8ex>[r]\ar@{.>}@<0.8ex>[l] & \cdots \ar@{.>}@<0.8ex>[l]\ar@{.>}@<0.8ex>[r] & \pi(b)-1  \ar@{.>}@<0.8ex>[l]\ar@{.>}@<0.8ex>[r] & \pi(b) \ar@{.>}@<0.8ex>[l]
}\,.
\end{equation}
% remark:tensor_in_steps.
The intermediate objects are $Z_i = S_{i, \pi(b)} \otimes_{A_{\pi(b)}} ( Y \l X )$ for $i < \pi(b)$. Each of these objects $Z_i$ is obtained from $Z_{i+1}$ by splitting an idempotent of the form $a^\dagger a$.

Suppose we are in the situation where $Y \circ X$ is identified by a universal property but an explicit object realising this universal property is not given as part of $\cat{B}$. Then the result of the $\pi(b)$-step computation depicted above is an explicit object $Z_0$ of $\cat{A}(a,c) \cong \cat{B}(a,c)$ with this universal property. Thus the cut system gives a process for constructing representing objects for the uniquely defined isomorphism class $Y \circ X$ of $\cat{B}(a,c)$.

Conceptually, the number of steps in \eqref{eq:computation_fiber_diagram} represents the computational cost of a cut across the object $b$. For any cut across this object, $\pi(b)$-steps are necessary to convert the ``uncertainty'' represented by $Y \l X$ with its $A_{\pi(b)}$-action to the certainty of an object of $\cat{A}(a,c)$ (in the sense of Remark \ref{remark:tensor_and_idempotent}). In the example of the next section, the cost $\pi(W)$ of cutting across a hypersurface singularity $W$ is determined by its dimension.

\section{The Landau-Ginzburg cut system}\label{section:lg_cut_system}

In this section we introduce the cut system $\cat{L}$, beginning with a sketch of the major pieces and then proceeding to the technical details. The objects of the cut system are the same as the objects of the bicategory $\LG$, that is, pairs $(x, W)$ consisting of a set of variables $x = (x_1,\ldots,x_n)$ and a potential $W \in k[x]$. Such an object is often written as $W$ or $W(x)$. The integer $\pi(x,W) = n$ is the number of variables.

Given objects $(x,W)$ and $(z,V)$ the supercategory $\cat{L}(W, V)$ is
\[
\cat{L}(W,V) = \hmf( k[x,z], V - W )^{\omega}\,,
\]
where $(-)^\omega$ denotes the idempotent completion; see Remark \ref{remark:idempotent_completion}. The objects of $\cat{L}(W,V)$ are therefore pairs $(X,e)$ consisting of a matrix factorisation $X$ of $V - W$ and an idempotent $e: X \lto X$.

Next we sketch the definition of the cut functors for $\cat{L}$. Let $W \in k[x_1,\ldots,x_n], V \in k[z_1,\ldots,z_m]$ and $U \in k[y_1,\ldots,y_p]$ be potentials. The cut functor
\begin{gather*}
\cat{L}(W, V) \otimes_k \cat{L}(V, U) \lto \cat{L}(W, U)^\bullet\,\\
X \otimes Y \mapsto Y \l X
\end{gather*}
of degree $m$ is defined on matrix factorisations $X$ of $V - W$ and $Y$ of $U - V$ as follows. The object $Y \l X$ is a finite rank matrix factorisation of $W - U$ equipped with an action of the Clifford algebra $A_m$. The underlying matrix factorisation is
\[
Y \l X = Y \otimes_{k[z]} J_V \otimes_{k[z]} X
\]
where
\[
J_V = k[z] / ( \partial_{z_1} V, \ldots, \partial_{z_m} V )\,.
\]
With the differential $d_Y \otimes 1 + 1 \otimes d_X$, $Y \l X$ is a finite rank matrix factorisation of $U - W$ over $k[x,y]$ by virtue of the fact that $J_V$ is a finite rank free $k$-module. As an explicit matrix of polynomials in $k[x,y]$ the differential on $Y \l X$ can be obtained from the differentials on $X$ and $Y$ (which involve $k[z,x]$ and $k[y,z]$ respectively) by replacing occurrences of $z$ by a certain square matrix of scalars; see the proof of Theorem \ref{theorem:l_cut_system} for more details.

It remains to define the action of the generators $a_i$ and $a_i^\dagger$ of $A_m$ on $Y \l X$. We give the formulas below; later we will prove that these operators satisfy Clifford relations and that, with this action, $X \otimes Y \mapsto Y \l X$ is a cut functor.

The key ingredients of the Clifford action are the Atiyah classes of $Y \otimes X$, which are defined by first taking the technical step of extending scalars
\begin{equation}\label{eq:completed_tensor_product}
Y \,\check{\otimes}\, X = Y \otimes_{k[z]} k\llbracket z \rrbracket \otimes_{k[z]} X
\end{equation}
and putting a connection on this module. We write $t = (t_1,\ldots,t_m)$ for the quasi-regular sequence of partial derivatives $t_i = \partial_{z_i} V$ and recall that there is by \cite[Appendix B]{dm1102.2957} a $k$-linear flat connection (which is ``standard'' in the sense of \cite[Definition 8.6]{dm1102.2957})
\begin{equation}\label{eq:connection_nabla}
\nabla: k\llbracket z \rrbracket \lto k\llbracket z \rrbracket \otimes_{k[t]} \Omega^1_{k[t]/k}
\end{equation}
the components of which are $k$-linear operators denoted $\partial_{t_i}: k\llbracket z \rrbracket \lto k\llbracket z \rrbracket$. A simple example of such a connection is given in Section \ref{example:computing_homs}. The operators $\partial_{t_i}$ extend to $k[x,y]$-linear operators on the free module $Y \, \check{\otimes}\, X$ and the $i$th Atiyah class is the operator
\[
\At_i = [d_{Y \otimes X}, \partial_{t_i}]\,.
\]
on $Y \,\check{\otimes}\, X$. This operator is the $i$th Atiyah class of $Y \,\check{\otimes}\, X$ relative to the ring morphism $k[x,y] \lto k[x,y,t]$ as defined in \cite[Section 9]{dm1102.2957}. A reference for Atiyah classes is \cite{buchweitz_flenner}.

The Atiyah class is $k[t]$-linear and passes to a $k$-linear operator on the quotient $Y \l X$ of $Y \,\check{\otimes}\, X$. We need another operator on this matrix factorisation, for $1 \le j \le m$
\begin{equation}\label{eq:lambda_defn}
\lambda_j = 1 \otimes \partial_{z_j}(d_X)\,.
\end{equation}
With this notation, the action of the generators $a_i, a_i^\dagger$ of $A_m$ on $Y \l X$ is defined by
\begin{equation}\label{eq:intro_clifford_act1}
a_i^\dagger = \At_i
\end{equation}
and
\begin{equation}\label{eq:intro_clifford_act2}
a_i = - \lambda_i - \sum_{l \ge 1} \sum_{q_1,\ldots,q_l} \frac{1}{(l+1)!} [ \lambda_{q_l}, [ \lambda_{q_{l-1}}, [ \cdots \big[ \lambda_{q_1}, \lambda_i ] \cdots ] \At_{q_1} \cdots \At_{q_l}
\end{equation}
where $1 \le q_i \le m$ for $1 \le i \le l$. For explicit examples of these operators, see Section \ref{example:computing_homs}.

The remaining components of the cut system $\cat{L}$ are the units $\Delta_W \in \cat{L}(W,W)$, which are the same as those for $\LG$ given in Section \ref{section:superbicatLG}, and the associators and unitors. To define the associator, let $W, V, U, Q$ be potentials with $V \in k[z_1,\ldots,z_m]$ and $U \in k[y_1,\ldots,y_p]$ and suppose we are given a triple of composable $1$-morphisms
\begin{equation}\label{eq:composable_triple}
\xymatrix{
(q, Q) & (y, U) \ar[l]_-{Z} & (z,V) \ar[l]_-{Y} & (x,W) \ar[l]_-{X}
}\,.
\end{equation}
The isomorphism of matrix factorisations
\begin{gather*}
\alpha: Z \l (Y \l X) = Z \otimes J_U \otimes ( Y \otimes J_V \otimes X ) \lto (Z \otimes J_U \otimes Y) \otimes J_V \otimes X = (Z \l Y) \l X,\\
z \otimes (y \otimes x) \mapsto (z \otimes y) \otimes x
\end{gather*}
turns out to be $A_m \otimes A_p$-linear, and defines the associator for $\cat{L}$.

The unitors can also be described by explicit formulas. Given $X \in \cat{L}(W,V)$ as above with homogeneous basis $e_i$ as a $\mathbb{Z}_2$-graded $k[x,z]$-module, we define morphisms of matrix factorisations
\begin{gather}
\mu_0: X \lto \Delta_V \l X \,, \\
\mu_0(e_i) = \sum_{l \ge 0} \sum_{i_1 < \cdots < i_l} \sum_j \Theta_{i_1} \ldots \Theta_{i_l} \left\{ \partial^{z,z'}_{[i_l]}(d_X) \cdots \partial^{z,z'}_{[i_1]}(d_X) \right\}_{ji} \otimes e_j\label{eq:explicit_mu0}
\end{gather}
and
\begin{gather}
\eta_0: X \lto X \l \Delta_W \,,\\
\eta_0(e_i) = \sum_{l \ge 0} \sum_{i_1 < \cdots < i_l} \sum_j (-1)^{\binom{l}{2} + l|e_i|} e_j \otimes \left\{ \partial^{x,x'}_{[i_1]}(d_X) \cdots \partial^{x,x'}_{[i_l]}(d_X) \right\}_{ji} \Theta_{i_1} \ldots \Theta_{i_l}
\end{gather}
To make sense of these formulas, note that $\Delta_V$ is a matrix factorisation of $V(z) - V(z')$ over $k[z,z']$ and $\Delta_W$ is a matrix factorisation of $W(x) - W(x')$ over $k[x,x']$, while the divided difference operators are defined by
\[
{}^{t_i}(-): k[x,x'] \lto k[x,x']\,, \qquad f \mapsto f|_{x_i \mapsto x'_i}
\]
and
\[
\partial^{x,x'}_{[i]}: k[x,x'] \lto k[x,x']\,, \qquad f \mapsto \frac{{}^{t_1 \ldots t_{i-1}} f - {}^{t_1 \ldots t_i} f}{x_i - x'_i}
\]
and in the same way for $z,z'$. For a concrete example of these formulas see \cite{lgdual_survey}. The actual unitors are the morphisms $X \lto \Delta_V \l X$ and $X \lto X \l \Delta_W$ in $\cat{L}(W,V)^\bullet$ corresponding to $\mu_0$ and $\eta_0$ under the bijection of Lemma \ref{lemma:morphism_out_1}. It is not meant to be clear \emph{a priori} that these formulas even define morphisms, let alone that they satisfy the axioms for unitors; all these checks will be performed below.
\\

The major theorem, to be proved over the course of this section:

\begin{theorem}\label{theorem:l_cut_system} $\cat{L}$ is a cut system.
\end{theorem}

\subsection{The details}

We present the cut $Y \l X$ with its Clifford action as arising from the following problem: how do we find a finite model of the matrix factorisation $Y \otimes X$ over $k[x,y]$? This problem was addressed in joint work with Dyckerhoff \cite{dm1102.2957} and we rely heavily on these results; the main new ingredient here is the construction of Appendix \ref{section:cliffordactkos}.

The solution is given in the language of homological perturbation, which is recalled in Appendix \ref{section:homolog_fix}, with a reformulation in terms of splitting homotopies due to Barnes and Lambe \cite{barneslambe}. From the point of view of perturbation it is natural to look for a decomposition $d_{Y \otimes X} = d + \delta$ where a finite representative of $(Y \otimes X, d)$. Then the perturbation lemma can be used to parlay this into a finite model for the object of interest $(Y \otimes X, d + \delta)$. 

There seems to be no obvious way to do this, but there is a simple way out: we replace $Y \otimes X$ by a sum of copies
\begin{equation}\label{eq:larger_object}
(Y \otimes X) \oplus (Y \otimes X)[1] \oplus (Y \otimes X) \oplus \cdots
\end{equation}
in such a way that this larger object \emph{does} fit into the pattern of the perturbation lemma. More precisely, \eqref{eq:larger_object} will be the tensor product of $Y \otimes X$ with a $\mathbb{Z}_2$-graded $k$-module and we find in \eqref{eq:iso_larger_object} an isomorphic object for which a finite model may be found by perturbing a finite model for one ``piece'' of the differential. Then we need only remember the data of how to extract $Y \otimes X$ from inside this larger object: this data is the Clifford action.
\medskip

% As long as the $t_i$ belong to the maximal ideal
Let us now proceed to the details. We keep the above notation, so $k[z] = k[z_1,\ldots,z_m]$. The Koszul complex $K$ of the sequence $t_1,\ldots,t_m$ (where $t_i = \partial_{z_i} V$) has as its underlying graded module the exterior algebra on the $\theta_i$ of degree $-1$. If $F = k\theta_1 \oplus \cdots \oplus k\theta_m$ then
\begin{equation}\label{defn:koszul}
K = \bigwedge F \otimes_k k[z], \qquad d_K = \sum_{i=1}^m t_i \theta_i^*\,.
\end{equation}
As explained in Appendix \ref{section:cliffordactkos} since each $t_i$ acts null-homotopically on $Y \otimes X$ there is an isomorphism of linear factorisations (writing $S_m = \bigwedge F$ as per our usual convention)
\begin{equation}\label{eq:iso_larger_object}
\xymatrix@C+3pc{ K \otimes Y \otimes X \ar@<-1ex>[r]_-{ \exp(\delta) } & S_m \otimes_k (Y \otimes X)\ar@<-1ex>[l]_-{ \exp(-\delta) } }
\end{equation}
where the differential on the left is $d_{Y \otimes X} + d_K$, on the right the differential is $d_{Y \otimes X}$, and $\delta = \sum_i \lambda_i \theta_i^*$ where $\lambda_i$ is a choice of null-homotopy for the action of $t_i$. 

From the identity $d^2_Y = V - W$ we deduce that
\[
\partial_{z_i}(d_Y) d_Y + d_Y \partial_{z_i}(d_Y) = \partial_{z_i} V
\]
so that we may choose $\lambda_j$ to be the operator \eqref{eq:lambda_defn}. Note that the right hand side of \eqref{eq:iso_larger_object} is a direct sum of (shifted) copies of $Y \otimes X$; this is the object we intended in \eqref{eq:larger_object}. The Clifford algebra $A_m$ naturally acts on this matrix factorisation via closed linear operators.

The next step is to find a finite model of $K \otimes Y \otimes X$ using the perturbation lemma, by writing the differential as $d_K + d_{Y \otimes X}$ and finding a finite model of $(K \otimes Y \otimes X, d_K)$. We begin by giving a finite model for the Koszul complex $(K, d_K)$ using connections.
\medskip

The connection $\nabla$ of \eqref{eq:connection_nabla} extends canonically to an operator on $k\llbracket z \rrbracket \otimes_{k[t]} \Omega^*_{k[t]/k}$ where $\Omega^* = \bigwedge \Omega^1$ is the exterior algebra. By identifying $d t_i$ with $\theta_i$ we may identify this tensor product with $K$ itself, so that $\nabla$ becomes identified with a degree $-1$ $k$-linear operator on $K$.  It is straightforward to check that $[d_K, \nabla]$ is invertible on the graded submodule $K_{\le -1}$ of nonzero degree terms, and we define
\[
H = [d_K, \nabla]^{-1} \nabla\,.
\]

\begin{proposition} $H$ is a $k$-linear splitting homotopy on the complex $K$.
\end{proposition}
\begin{proof}
See \cite[Section 8.1]{dm1102.2957}.
\end{proof}

The associated strong deformation retract of $\mathbb{Z}$-graded $k$-complexes is
\begin{equation}\label{eq:originalHdef}
\xymatrix@C+2pc{
(J_V,0) \ar@<-1ex>[r]_{\sigma} & (K, d_K), \ar@<-1ex>[l]_{\pi}
} \quad H
\end{equation}
where $\pi: K \lto J_V$ is the canonical quasi-isomorphism which vanishes on $K_{\le -1}$ and is in degree zero the quotient map $k\llbracket z \rrbracket \lto J_V$, and $\sigma$ is a $k$-linear embedding of $J_V$ into $k\llbracket z \rrbracket$ uniquely determined by $\nabla$. The splitting homotopy $H$, or equivalently the deformation retract \eqref{eq:originalHdef}, is the desired finite model of the Koszul complex $K$.

Recall that the tensor product $Y \,\check{\otimes}\, X$ of \eqref{eq:completed_tensor_product} is a finite rank matrix factorisation of $U - W$ over $R = k[x,y]\llbracket z \rrbracket$. Let us write $\ell = k[x,y]$ so that the connection $\nabla$ extends to a standard $\ell$-linear flat connection on $R$. A choice of homogeneous basis for $X$ and $Y$ allows us to write $Y \otimes X \cong k[x,y,z] \otimes_k V$ for some $\mathbb{Z}_2$-graded free $k$-module $V$, so
\begin{equation}\label{eq:lambdaiso2}
K \otimes_{k\llbracket z \rrbracket}( Y \,\check{\otimes}\, X ) \cong K \otimes_{k\llbracket z \rrbracket} (R \otimes_k V) \cong K \otimes_k k[x,y] \otimes_k V
\end{equation}
In this way the $k$-linear splitting homotopy $H$ on $K$ induces an $\ell$-linear operator $H \otimes 1$ on $K \otimes_{k\llbracket z \rrbracket} Y \,\check{\otimes}\, X$, which we again denote by $H$.

\begin{lemma} $H$ is an $\ell$-linear splitting homotopy on the complex $(K \otimes_{k\llbracket z \rrbracket} Y \,\check{\otimes}\, X, d_K)$.
\end{lemma}

The associated strong deformation retract is
\[
\xymatrix@C+2pc{
(Y \l X,0) \ar@<-1ex>[r]_-{\sigma} & (K \otimes_{k\llbracket z \rrbracket} Y \,\check{\otimes}\, X, d_K), \ar@<-1ex>[l]_-{\pi}
} \quad H
\]
The map $\pi$ is $R$-linear induced by the map $\pi: K \lto J_V$, and $\sigma = 1 \otimes \sigma$ is defined using map $\sigma$ of \eqref{eq:originalHdef} and the isomorphism \eqref{eq:lambdaiso2}. This is the desired finite model of $K \otimes Y \otimes X$ equipped with only the differential $d_K$. 

We now view $d_{Y \otimes X}$ as a perturbation. The transference problem asks how to define a splitting homotopy $\phi_\infty$ on the matrix factorisation $(K \otimes_{k\llbracket z \rrbracket} Y \,\check{\otimes}\, X , d_K + d_{Y \otimes X})$ in such a way that the underlying graded module is $Y \l X$. Since $H d_{Y \otimes X}$ has finite order, it follows from the perturbation lemma (Theorem \ref{theorem:pertlemma}) that
\[
\phi_\infty = \sum_{m \ge 0} (-1)^m (H d_{Y \otimes X})^m H
\]
is an $\ell$-linear splitting homotopy achieving the desired aim.

Specifically, by \cite[Proposition 7.1]{dm1102.2957} $\phi_\infty$ is an $\ell$-linear splitting homotopy on $K \otimes_{k\llbracket z \rrbracket} Y \,\check{\otimes}\, X$ and the associated $\ell$-linear strong deformation retract of linear factorisations of $U - W$ is of the form
\begin{equation}\label{eq:finite_model_defo}
\xymatrix@C+4pc{
Y \l X \ar@<-1ex>[r]_-{\sigma_\infty} & K \otimes_{k\llbracket z \rrbracket} Y \,\check{\otimes}\, X, \ar@<-1ex>[l]_-{\pi}
} \quad \phi_\infty
\end{equation}
where the differential on the left is $d_{Y \otimes X}$ and on the right it is $d_K + d_{Y \otimes X}$. 
%$A = d_{Y \otimes X}( 1 + H d_{Y \otimes X} )^{-1}$ and $\sigma_\infty = \sigma - H A \sigma$. We need only justify why in \eqref{eq:perturbedsdr} we have $d_\infty = d_{Y \otimes X}$ and $f_\infty = \pi$. But this follows from $H \sigma = 0$ and $\pi H = 0$. 
%\begin{equation}
%\xymatrix@C+4pc{
%Y \l X \ar@<-1ex>[r]_-{\exp(-\delta) \sigma_\infty} & S_m \otimes_k ( Y \,\check{\otimes}\, X )\ar@<-1ex>[l]_-{\pi \exp(\delta)}
%}\,.
%\end{equation}
%of linear factorisations of $U - W$ over $k[x,y]$. 

\begin{proposition}\label{prop:final_homotopy_equiv_cut} There is a homotopy equivalence of matrix factorisations over $k[x,y]$
\begin{equation}\label{eq:final_finite_model}
\xymatrix@C+4pc{
Y \l X \ar@<-1ex>[r]_-{\Phi^{-1}} & S_m \otimes_k ( Y \otimes X )\ar@<-1ex>[l]_-{\Phi}
}
\end{equation}
where $\Phi = \pi \exp(-\delta) \epsilon$ and $\Phi^{-1} = \epsilon^{-1} \exp(\delta) \sigma_\infty$ and $\epsilon: Y \otimes X \lto Y \,\check{\otimes}\, X$ is canonical.
\end{proposition}
\begin{proof}
Together \eqref{eq:iso_larger_object} and \eqref{eq:finite_model_defo} give a homotopy equivalence
\begin{equation}\label{eq:final_finite_model2}
\xymatrix@C+4pc{
Y \l X \ar@<-1ex>[r]_-{\sigma_\infty} & K \otimes Y \,\check{\otimes}\, X \ar@<-1ex>[l]_-{\pi} \ar@<-1ex>[r]_-{ \exp(\delta) } & S_m \otimes_k ( Y \,\check{\otimes}\, X ) \ar@<-1ex>[l]_-{ \exp(-\delta) }
}\,.
\end{equation}
By \cite[Remark 7.7]{dm1102.2957} $\epsilon$ is a homotopy equivalence over $k[x,y]$, completing the proof.
\end{proof}

This achieves our aim: we took direct sums of copies of $Y \otimes X$ to form $S_m \otimes_k( Y \otimes X )$ and then used perturbation to find a finite model of this latter object. It is worth noting that, in the language of splitting homotopies, the cut $Y \l X$ is the solution of a fixed point problem formulated in terms of linear operators on the initial data $S_m \otimes_k( Y \otimes X )$. 

This equivalence induces an $A_m$ action on $Y \l X$ via the action on $S_m$.

\begin{definition} We introduce closed odd $k[x,y]$-linear operators on $Y \l X$ by
\begin{equation}\label{eq:new_operators}
(a_i^\dagger)_{Y,X} = \Phi \circ a_i^\dagger \circ \Phi^{-1}, \qquad (a_i)_{Y,X} = \Phi \circ a_i \circ \Phi^{-1}\,.
\end{equation}
\end{definition}

\begin{proposition}\label{prop:clifford_action} The operators $(a_i)_{Y,X}, (a_i^{\dagger})_{Y,X}$ of \eqref{eq:new_operators} agree with those in \eqref{eq:intro_clifford_act1}, \eqref{eq:intro_clifford_act2}.
\end{proposition}
\begin{proof}
Let us define intermediate operators on $K \otimes Y \otimes X$
\[
\psi_i^\dagger = \exp(-\delta) a_i^\dagger \exp(\delta), \qquad \psi_i = \exp(-\delta) a_i \exp(\delta).
\]
By Appendix \ref{section:cliffordactkos}, $\psi_i^\dagger = \theta^*_i$ and
\[
\psi_i = \theta_i - \lambda_i - \sum_{l \ge 1} \sum_{q_1,\ldots,q_l} \frac{1}{(l+1)!} [ \lambda_{q_l}, [ \lambda_{q_{l-1}}, [ \cdots \big[ \lambda_{q_1}, \lambda_i ] \cdots ] \theta_{q_1}^* \cdots \theta_{q_l}^*\,.
\]
Next we recall that, with $t_i = \partial_{z_i} V$, we have by \cite[Section 10]{dm1102.2957}
\begin{equation}\label{eq:atiyah_formula_sigma}
\sigma_\infty = \sum_{l \ge 0} \sum_{q_1,\ldots,q_l} (-1)^{\binom{l+1}{2}} \frac{1}{l!} [ d_{Y \otimes X}, \partial_{t_{q_1}}] \cdots [ d_{Y \otimes X}, \partial_{t_{q_l}}] \theta_{q_1} \cdots \theta_{q_l} + (t \text{ terms})\,.
\end{equation}
That is, $\sigma_\infty$ is given modulo the submodule $(t_1,\ldots,t_m) K \otimes Y \otimes X$ by the formula above. Hence we compute
\begin{align*}
(a_i^\dagger)_{Y,X} &= \pi \psi_i^\dagger \sigma_\infty = \pi \theta_i^* \sigma_\infty\\
&= \sum_{l \ge 0} \sum_{q_1,\ldots,q_l} (-1)^{\binom{l+1}{2}} \frac{1}{l!} [ d_{Y \otimes X}, \partial_{t_{q_1}}] \cdots [ d_{Y \otimes X}, \partial_{t_{q_l}}] \pi \theta_i^* \theta_{q_1} \cdots \theta_{q_l}\\
&= \sum_q [d_{Y \otimes X}, \partial_{t_q}] \pi \theta_i^* \theta_q\\
&= [d_{Y \otimes X}, \partial_{t_i}] = \At_i\,.
\end{align*}
To simplify the notation in the following, we define ($i$ is fixed) for a tuple $\bold{q} = (q_1,\ldots,q_l)$
\begin{gather*}
\tau_{\bold{q}} = [ \lambda_{q_l}, [ \lambda_{q_{l-1}}, \ldots [ \lambda_{q_1}, \lambda_i ] \cdots ]\,,\\
\At_{\bold{q}} = [ d_{Y \otimes X}, \partial_{t_{q_1}}] \cdots [ d_{Y \otimes X}, \partial_{t_{q_l}}]\,.
\end{gather*}
With this notation we compute
\begin{align*}
(a_i)_{Y,X} &= \pi \psi_i \sigma_\infty\\
&= - \lambda_i - \pi \sum_{l \ge 1, k \ge 0} \sum_{q_1,\ldots,q_l} \sum_{w_1,\ldots,w_k} \frac{1}{(l+1)! k!} (-1)^{kl+\binom{k+1}{2}} \tau_{\bold{q}}  \At_{\bold{w}} \theta_{q_1}^* \cdots \theta_{q_l}^* \theta_{w_1} \cdots \theta_{w_k}\,.
\end{align*}
But $\pi$ is non-vanishing only on terms of $\theta$-degree zero, so the sum restricts to $l = k$ and to $\bold{q}$ a permutation of $\bold{w}$. Since
\[
\theta_{q_1}^* \cdots \theta_{q_l}^* \theta_{q_{\sigma(1)}} \cdots \theta_{q_{\sigma(l)}} = (-1)^{|\sigma|} (-1)^{\binom{l}{2}}
\]
we have, using the fact that the Atiyah classes anti-commute,
\begin{align*}
(a_i)_{Y,X} &= - \lambda_i - \sum_{l \ge 1} \sum_{q_1,\ldots,q_l} \sum_{\sigma \in S_l} (-1)^{|\sigma|} \frac{1}{l! (l+1)!} \tau_{\bold{q}} \At_{\sigma(\bold{q})}\\
&= - \lambda_i - \sum_{l \ge 1} \sum_{q_1,\ldots,q_l} \frac{1}{(l+1)!} \tau_{\bold{q}} \At_{\bold{q}}
\end{align*}
which completes the proof.
\end{proof}

%The upshot is that if we equip $Y \l X$ with the Clifford action $\{ a_i, a_i^\dagger \}_i$ defined at the beginning of this section then the homotopy equivalence of Proposition \ref{prop:final_homotopy_equiv_cut} is $A_m$-linear.

\begin{remark} There is a question of whether or not the $A_m$-action on $Y \l X$ is canonical, since it depends on the choice of connection $\nabla$ and on a choice of homogeneous basis for $X$ and $Y$ (for example, when we write $\partial_{z_j}(d_X)$ we implicitly choose a basis so as to represent $d_X$ as a matrix and differentiate its entries). But Atiyah classes are independent, up to homotopy, of the choice of connection, and it is easy to see that the operator $\partial_{z_j}(d_X)$ is similarly independent of the choice of basis. Hence the action is indeed canonical.
\end{remark}

\begin{proposition} The above construction defines a cut functor of degree $m$
\begin{gather*}
\cat{L}(W, V) \otimes_k \cat{L}(V,U) \lto \cat{L}(W, U)^\bullet\\
X \otimes Y \mapsto (Y \l X, \{ a_i, a_i^\dagger \}_i )
\end{gather*}
sending a morphism $\phi \otimes \psi$ to $\psi \l \phi = \psi \otimes 1 \otimes \phi$.
\end{proposition}
\begin{proof}
We check that the construction is functorial, the compatibility with the supercategory structure is straightforward and we omit it. We only check functoriality in $X$, as the check for $Y$ is the same. So let $\phi: X \lto X'$ be a morphism of matrix factorisations. Let $a_i, a_i^\dagger$ denote the actions on both $X$ and $X'$, so that we need to check
\[
(1\l\phi) a_i = a_i (1 \l \phi), \qquad (1\l\phi) a_i^\dagger = a_i^\dagger (1 \l \phi)\,.
\]
It is enough to prove that $(1 \l \phi) \At_i = \At_i (1 \l \phi)$ and $(1 \l \phi) \lambda_i = \lambda_i (1 \l \phi)$ up to homotopy. The former follows from naturality of the Atiyah class (the arguments in either \cite[Lemma 3.3]{buchweitz_flenner} or \cite[Lemma 3.6]{dm1102.2957} also apply in the current context to show the desired naturality). For the latter we use Lemma \ref{lemma:naturalityoflambda} below.
\end{proof}

\begin{remark}\label{remark:iso_quotient_map} If $\iota: k \lto S_m$ denotes the map $\iota(1) = 1$ then by Lemma \ref{lemma:morphism_out_1} the composite
\[
\xymatrix@C+2pc{
Y \otimes X \cong k \otimes_k ( Y \otimes X ) \ar[r]^-{\iota \otimes 1} & S_m \otimes_k ( Y \otimes X ) \ar[r]^-{\Phi} & Y \l X
}
\]
is simply the quotient map $\varrho: Y \otimes X \lto Y \l X$.
\end{remark}

\begin{lemma}\label{lemma:naturalityoflambda} Let $R$ be a $k$-algebra, $W \in R$. Let $\partial$ be a $k$-linear derivation on $R$ and $\phi: X \lto X'$ a morphism of finite rank matrix factorisations over $R$ with chosen homogeneous bases. Then the diagram
\[
\xymatrix{
X \ar[d]_{\partial(d_X)} \ar[r]^\phi & X' \ar[d]^{\partial(d_{X'})}\\
X \ar[r]_{\phi} & X'
}
\]
commutes up to homotopy.
\end{lemma}
\begin{proof}
From $d_{X'} \phi = \phi d_X$ we deduce
\[
\partial( d_{X'} ) \phi + d_{X'} \partial( \phi ) = \partial( \phi ) d_X + \phi \partial( d_X )
\]
and hence
\[
\partial( d_{X'} ) \phi - \phi \partial( d_X ) = \partial(\phi) d_X - d_{X'} \partial(\phi)
\]
as claimed.
\end{proof}

To complete the construction of the cut system we need to finish defining the associator and unitors (a task begun at the outset of the section) and to prove that the coherence conditions hold. Regarding the associator, we only need to check that $\alpha$ commutes with the Clifford actions. In the context of \eqref{eq:composable_triple},

\begin{lemma} The morphism $\alpha: (Z \l Y) \l X \lto Z \l (Y \l X)$ is $A_m \otimes A_p$-linear.
\end{lemma}
\begin{proof}
We prove the $A_p$-linearity of $\alpha$, the proof for $A_m$ is similar. The algebra $A_p$ acts on the cut between $Z$ and $Y$ involving the variables $y_1,\ldots,y_p$. The action on $Z \l ( Y \l X ) = Z \otimes J_U \otimes (Y \otimes J_V \otimes X)$ is via $1 \otimes (\partial_{y_i}(d_Y) \otimes 1)$ and the Atiyah classes of 
\[
M = Z \otimes_{k[y]} k\llbracket y \rrbracket \otimes_{k[y]} \big( Y \otimes_{k[z]} J_V \otimes_{k[z]} X \big)
\]
relative to the ring morphism $k[q,x] \lto k[q,x] [ \partial_{y_1} U, \ldots, \partial_{y_p} U ]$. Specifically, the generator $a_i^{\dagger}$ of $A_p$ is the commutator $[d_{M}, \partial_s]$ where $s = \partial_{y_i} U$. But this operator can be identified with an Atiyah class of the linear factorisation
\begin{equation}\label{eq:assoc_proof_M}
Z \otimes_{k[y]} k\llbracket y \rrbracket \otimes_{k[y]} \big( Y \otimes_{k[z]} k\llbracket z \rrbracket \otimes_{k[z]} X \big)
\end{equation}
relative to the ring morphism
\begin{equation}\label{eq:assoc_proof_M2}
k[q,x] \lto k[q,x] [ \partial_{y_1} U, \ldots, \partial_{y_p} U, \partial_{z_1} V, \ldots, \partial_{z_m} V ]\,.
\end{equation}
By the same argument the action of $A_p$ on $(Z \l Y ) \l X$ is via $(1 \otimes \partial_{y_i}(d_Y)) \otimes 1$ and the Atiyah classes of $\big( Z \otimes_{k[y]} k\llbracket y \rrbracket \otimes_{k[y]} Y \big) \otimes_{k[z]} k\llbracket z \rrbracket \otimes_{k[z]} X$ relative to the same ring morphism \eqref{eq:assoc_proof_M2}. It is therefore clear that the associator $\alpha$ identifies these two actions.
\end{proof}

Earlier the unitors were described by explicit formulas. Here we begin more abstractly: let $X \in \cat{L}(W,V)$ be a matrix factorisation with $W \in k[x_1,\ldots,x_n]$ and $V \in k[z_1,\ldots,z_m]$. Since $\Delta_V$ is the unit in $\LG$ there is a homotopy equivalence $\lambda: \Delta_V \otimes_{k[z]} X \lto X$ and so by Proposition \ref{prop:final_homotopy_equiv_cut} there are homotopy equivalences
\begin{equation}
\xymatrix@C+3pc{
\Delta_V \l X \ar@<-1ex>[r]_-{\Phi^{-1}} & S_m \otimes_k ( \Delta_V \otimes X )\ar@<-1ex>[l]_-{\Phi} \ar@<-1ex>[r]_-{1 \otimes \lambda} & S_m \otimes_k X \ar@<-1ex>[l]_-{1 \otimes \lambda^{-1}} 
}\,.
\end{equation}
By construction these isomorphisms are $A_m$-linear, and the map from right to left is an isomorphism $\mu: X \lto \Delta_V \l X$ in $\cat{L}(W,V)^\bullet$ corresponding under Lemma \ref{lemma:morphism_out_1} to
\[
\mu_0 = \xymatrix@C+1pc{
X \ar[r]^-{\lambda^{-1}} & \Delta_V \otimes_{k[z]} X \ar[r]^-{\varrho} & \Delta_V \l X
}\,.
\]
where $\varrho$ is the quotient map. Given the explicit formula for $\lambda^{-1}$ from \cite[Section 4.1]{lgdual}, this map $\mu_0$ has the explicit formula given earlier in \eqref{eq:explicit_mu0}. Since $\mu_0$ is natural in $X$, this also proves that $\mu$ is a natural isomorphism. The details for $\eta$ are similar, so the unitors are now defined.

\begin{remark}\label{remark:inflation} Let $V \in k[z_1,\ldots,z_m]$ be a potential and choose a $k$-basis for $J_V$. Then multiplication by $z_i$, as a $k$-linear operator $J_V \lto J_V$, gives a matrix $[z_i]$ over $k$. 

Given $X$ and $Y$ the cut $Y \l X$ has a differential which, as a matrix, can be described by taking the matrix of $d_Y \otimes 1 + 1 \otimes d_X$ over $k[x,y,z]$ and replacing every occurrence of $z_i$ by the scalar matrix $[z_i]$. This ``inflation'' produces a matrix of polynomials over $k[x,y]$. 
\end{remark}

\begin{proof}[Proof of Theorem \ref{theorem:l_cut_system}] We first check the coherence axioms. The first and third are trivial. To prove (2) we need by Lemma \ref{lemma:unitor_check_help} only check commutativity of \eqref{eq:unitor_check_help1} in the usual situation of $X: W \lto V$ and $Y: V \lto U$. 

This amounts to checking commutativity of
\begin{equation}\label{eq:coherence3_1}
\xymatrix{
(Y \l \Delta_V ) \l X \ar[rr]^-{\alpha} & & Y \l ( \Delta_V \l X )\\
(Y \otimes \Delta_V) \otimes J_V \otimes X \ar[u]^-{\varrho \otimes 1} & & Y \otimes J_V \otimes ( \Delta_V \otimes X) \ar[u]_-{1 \otimes \varrho}\\
Y \otimes J_V \otimes X \ar[u]^-{ \rho^{-1} \otimes 1} & Y \otimes J_V \otimes X \ar[l]^-{e_m} \ar[r]_-{e_m} & Y \otimes J_V \otimes X \ar[u]_-{1 \otimes \lambda^{-1}}
}\,.
\end{equation}
By definition $e_m$ factors as
\[
\xymatrix{
Y \l X \ar[r]^-{\Phi^{-1}} & S_m \otimes_k ( Y \otimes X ) \ar[r] & Y \otimes X \ar[r] & S_m \otimes_k (Y \otimes X) \ar[r]^-{\Phi} & Y \l X
}
\]
so commutativity of \eqref{eq:coherence3_1} follows from commutativity of
\[
\xymatrix@C-0.5pc{
(Y \l \Delta_V ) \l X \ar[rrr]^-{\alpha} & & & Y \l ( \Delta_V \l X )\\
(Y \otimes \Delta_V) \otimes J_V \otimes X \ar[u]^-{\varrho \otimes 1} & (Y \otimes \Delta_V ) \otimes X \ar[l]_-{\varrho} \ar[r]^-{\alpha} & Y \otimes (\Delta_V \otimes X) \ar[r]^-{\varrho} & Y \otimes J_V \otimes ( \Delta_V \otimes X) \ar[u]_-{1 \otimes \varrho}\\
Y \otimes J_V \otimes X \ar[u]^-{ \rho^{-1} \otimes 1} & Y \otimes X \ar[u]^-{\rho^{-1} \otimes 1} \ar@{=}[r] \ar[l]^-{\varrho} & Y \otimes X \ar[u]_-{1 \otimes \lambda^{-1}} \ar[r]_-{\varrho} & Y \otimes J_V \otimes X \ar[u]_-{1 \otimes \lambda^{-1}}
}\,.
\]
The innermost square of this diagram is the coherence constraint for the unitors in $\LG$.

Finally, we check the computability conditions. The associator and unitors are derived explicitly from $d_X$ and $d_Y$, so (3) is clear, and (2) is addressed by Remark \ref{remark:inflation} and the explicit formulas for the Clifford action. For (1), the objects and morphisms of $\cat{L}(W,V)$ are described by finite matrices of polynomials, and the composition rule is given by multiplication of matrices. There is one subtlety: morphisms are matrices of polynomials \emph{identified up to homotopy}. 

To argue that $\cat{L}(W,V)$ is effectively computable we need to show that it is possible to determine algorithmically whether two morphisms are homotopic. This is important even at the level of objects: an object is a pair $(X,e)$ where $e^2 = e$ holds up to homotopy. Without an effective homotopy relation, there would be no algorithmic way to even distinguish between objects and non-objects. The necessary check is in Section \ref{example:computing_homs}.
\end{proof}

\begin{remark}\label{remark:relation_to_toby_paper} The cut system presented here refines earlier work in \cite{dm1102.2957}. To see the connection, recall the idempotent $e_m \in A_m$ of Definition \ref{defn:idempotent_e} which corresponds to projection onto lowest degree in $S_m$. The projector onto top degree $k \cdot \theta_1 \cdots \theta_m \subset S_m$ is 
\[
e_m' = a_1 \cdots a_m a_m^\dagger \cdots a_1^\dagger\,.
\]
Using \eqref{eq:intro_clifford_act1} and \eqref{eq:intro_clifford_act2} we can write down the explicit form of this idempotent acting on $Y \l X$. On this object each of the $a_i$'s acts simply as $- \lambda_i$ because all of the $a_j^\dagger$'s appear to the right. Hence on $Y \l X$
\begin{align*}
e_m' &= (-1)^{\binom{m}{2}} a_1 \cdots a_m a_1^\dagger \cdots a_m^\dagger \\
&= (-1)^{\binom{m+1}{2}} \lambda_1 \cdots \lambda_m \At_1 \cdots \At_m\,.
\end{align*}
which is precisely the idempotent which appears in \cite[Corollary 10.4]{dm1102.2957}.

In \emph{ibid.} we constructed finite models of convolutions $Y \circ X$ one pair $(Y,X)$ at time. To make this assignment of finite models coherently for all $1$-morphisms in $\LG$ simultaneously it is natural to identify the Clifford actions which underlie the idempotents, as we do here.
\end{remark}

\subsection{The cut model of $\LG$}

Let $\cat{L}$ be the cut system described above, and let $\cat{L}^\bullet$ denote the associated superbicategory. With $\LG$ the superbicategory of Section \ref{section:superbicatLG} we construct a cut model $I: \LG \lto \cat{L}^\bullet$. On objects $I$ is the identity. For potentials $V,W$ we need to define a functor
\begin{equation}\label{eq:cut_model_equiv}
I_{V,W}: \LG(W,V) \lto \cat{L}^\bullet(W,V)\,.
\end{equation}
Recall that with $\cat{C} = \hmf(k[x,z], V - W)$,
\[
\cat{L}^\bullet(W,V) = \cat{C}^{\omega \bullet}\,, \quad \LG(W,V) = \cat{C}^{\oplus}\,.
\]
An object of $\cat{C}^{\omega \bullet}$ is a tuple $((X,e), n)$ consisting of an integer $n \ge 0$ and an $A_n$-module $(X,e)$ in $\cat{C}^\omega$, while an object of $\cat{C}^{\oplus}$ is a (possibly infinite) matrix factorisation of $V - W$ which is a direct summand of a finite rank one.

On the full subcategory of finite rank matrix factorisation of $V - W$, $I_{V,W}$ is the identity and this functor extends uniquely (up to natural equivalence) to $\LG(W,V)$. To be more explicit about it: for each $X \in \LG(W,V)$ we \emph{choose} a finite rank matrix factorisation $\bar{X}$ together with an idempotent $e$ and morphisms $f_X, g_X$ as in the diagram
\[
\xymatrix@C+2pc{
\bar{X} \ar@(ul,dl)_-{e} \ar@<1ex>[r]^-{f_X} & X \ar@<1ex>[l]^-{g_X}
}
\]
satisfying $g_X f_X = e$ and $f_X g_X = 1_X$. Then the full definition of $I$ is given by
\[
I( X ) = ((\bar{X}, e), 0)\,, \qquad I( \phi ) = g_Y \phi f_X\,.
\]
To complete the definition of a strong functor $I$ it remains to give a natural $2$-isomorphism
\begin{equation}\label{eq:strong_functor_varphi}
\varphi: I(Y \otimes X) \lto I(Y) \l I(X)
\end{equation}
for every pair of $1$-morphisms $X: W \lto V$ and $Y: V \lto U$ in $\LG$. 

We give the definition of $\varphi$ where $X,Y$ are finite rank matrix factorisations; the general case follows. Recall the $A_m$-linear isomorphism of Proposition \ref{prop:final_homotopy_equiv_cut}
\begin{equation}\label{eq:cutmodelequiv}
\xymatrix@C+4pc{
Y \l X \ar@<-1ex>[r]_-{\Phi^{-1}} & S_m \otimes_k ( Y \otimes X )\ar@<-1ex>[l]_-{\Phi}
}\,.
\end{equation}
% TODO issue with completion
This is essentially the desired natural isomorphism $\varphi$, and it is only necessary to package it a little more carefully. Let $e = e_m \in A_m$ denote the idempotent from Definition \ref{defn:idempotent_e}. If we define $\nu: S_m \lto k$ to be the projection onto $k \cdot 1$ then there is a diagram
\[
\xymatrix@C+4pc{
Y \l X \ar@<1ex>[r]^-{\nu \Phi^{-1}} & Y \otimes X \ar@<1ex>[l]^-{\varrho}
}
\]
expressing $Y \otimes X$ as the splitting of the idempotent $e$ on $Y \l X$. Here $\varrho: Y \otimes X \lto Y \l X$ is the quotient map. The usphot is that as objects of $\cat{L}(W,V)^{\omega \bullet}$
\[
I(Y \otimes X) = ((Y\l X, e), 0), \qquad I(Y) \l I(X) = ( (Y \l X, 1), m )\,.
\]
Morphisms $\varphi$ as in \eqref{eq:strong_functor_varphi} are in bijection with morphisms
\[
\varphi_0: (Y \l X, e) \lto (Y\l X, 1)
\]
in $\hmf(k[x,z], U - W)^{\omega}$ satisfying $\theta_i^* \circ \varphi_0 = 0$ for $1 \le i \le m$. The endomorphism $\varphi_0 = e_m$ of $Y \l X$ fits the bill. In the category of infinite rank matrix factorisations, $\varphi_0$ is just the morphism $\varrho$. It follows that $\varphi$ is a natural isomorphism, since it corresponds to $\Phi$.

\begin{theorem} The data $(I, \varphi)$ defines a cut model $\LG \lto \cat{L}^\bullet$.
\end{theorem}
\begin{proof}
Once we have shown that $(I, \varphi)$ is a strong functor it is clearly ``super'', and by construction it defines a cut model. There are two coherence conditions: the first is commutativity of the following diagram, in the context of a composable triple as in \eqref{eq:composable_triple}:
\begin{equation}\label{eq:dia_coherence1}
\xymatrix@C+2pc{
(I(Z) \l I(Y) ) \l I(X) \ar[r]^-{\alpha} \ar[d]_-{\varphi \l 1} & I(Z) \l ( I(Y) \l I(X) ) \ar[d]^-{1 \l \varphi} \\
I(Z \otimes Y) \l I(X) \ar[d]_-{\varphi} &  I(Z) \l I(Y \otimes X)\ar[d]^-{\varphi}\\
I( (Z \otimes Y) \otimes X ) \ar[r]_-{I(\alpha)} & I(Z \otimes (Y \otimes X))
}
\end{equation}
The key to checking this lies in understanding what $I( Z \otimes (Y \otimes X) )$ is. But as we have argued above, $Y \otimes X$ is a summand of $Y \l X$ with idempotent $e_m$ while $Z \otimes ( Y \l X )$ is a summand of $Z \l (Y \l X)$ with idempotent $e_p$. This implies that $Z \otimes (Y \otimes X)$ is a summand of $Z \l (Y \l X)$ with idempotent $e_p e_m = e_m e_p = e_{m+p}$, so
\[
I( (Z \otimes Y) \otimes X ) = ( Z \l (Y \l X), e_{m+p} ), \quad I( Z \otimes (Y \otimes X) ) = ( (Z \l Y) \l X, e_{m+p} )\,.
\]
Then commutativity of \eqref{eq:dia_coherence1} reduces to commutativity in $\hmf(Q - W)^{\omega}$ of
\[
\xymatrix@C+2pc{
( (Z \l Y ) \l X, 1 ) \ar[r]^-{\alpha} & ( Z \l (Y \l X ), 1)\\
( (Z \l Y) \l X, e_p \l 1) \ar[u]^-{e_p \l 1} & ( Z \l (Y \l X), 1 \l e_m) \ar[u]_-{e_m}\\
( (Z \l Y) \l X, e_{m+p} ) \ar[u]^-{e_{m+p}} \ar[r]_-{e_{m+p}} & ( Z \l (Y \l X), e_{m+p} ) \ar[u]_-{e_{m+p}}
}
\]
which is clear.

The remaining coherence conditions are commutativity, for $X: W \lto V$, of
\begin{equation}\label{eq:dia_coherence2}
\xymatrix{
I(X) \ar[dr]_-{\eta} & & I( X \otimes \Delta_W ) \ar[ll]_-{I(\rho)} \\
& I(X) \l \Delta_W \ar[ur]_-{\varphi}
}
\end{equation}
and a similar diagram for $\lambda$. By definition $I(\rho)$ is the inverse of
\[
\xymatrix{
X \ar[r]^-{\eta^{-1}} & X \otimes \Delta_W \ar[r]^{\varrho} & X \l \Delta_W
}
\]
so checking commutativity of \eqref{eq:dia_coherence2} can be done using Lemma \ref{lemma:morphism_out_1}, and similarly for the diagram involving $\mu$.
\end{proof}

\subsection{Example: the Hom complex}\label{example:computing_homs}

Using the cut model $\cat{L}$ we can study the complex $\Hom_R(X,Y)$ for two finite rank matrix factorisations $X,Y$ of a potential $V \in R = k[z_1,\ldots,z_m]$. In particular, we will explain how the homotopy relation on morphisms of matrix factorisations is effectively computable.

The pair $(k,0)$ is an object in $\LG$ and we may view $Y$ and the dual matrix factorisation $X^{\vee}$ of $-V$ as a pair of $1$-morphisms
\begin{equation}\label{eq:compute_hom_composition}
\xymatrix@C+2pc{
0 \ar[r]^-{Y} & V \ar[r]^-{X^{\vee}} & 0
}\,.
\end{equation}
The composite is a $1$-morphism $0 \lto 0$, that is, $\mathbb{Z}_2$-graded complex of $k$-modules. This is the Hom complex of $R$-linear homogeneous maps $X \lto Y$,
\[
X^{\vee} \otimes_R Y \cong \Hom_R(X, Y)
\]
with the differential $\phi \mapsto d_Y \circ \phi - (-1)^{|\phi|} \phi \circ d_X$. The cut model provides a two-step process to compute the result of the composition in \eqref{eq:compute_hom_composition}. First form the cut
\[
X^{\vee} \l Y = X^{\vee} \otimes_R J_V \otimes_R Y = \Hom_{J_V}( \bar{X}, \bar{Y} )
\]
where $\bar{X} = X \otimes_R J_V, \bar{Y} = Y \otimes_R J_V$. This is a complex of finite-rank $\mathbb{Z}_2$-graded free $k$-modules, and there is an $A_m$-linear homotopy equivalence
\[
\Phi: S_m \otimes_k \Hom_R(X,Y) \lto \Hom_{J_V}( \bar{X}, \bar{Y} )\,.
\]
Set $H = \Hom_{J_V}( \bar{X}, \bar{Y} )$ and define
\[
B = \im( d^1_H: H^1 \lto H^0 )\,.
\]
As a matrix of scalars $d^1_H$ may be computed from the differentials $d_X, d_Y$ by inflation, as described in Remark \ref{remark:inflation}. Let $\psi: X \lto Y$ be a morphism of matrix factorisations. Then $\psi$ is null-homotopic if and only if $\psi' = \Phi(\psi) \in \Hom_{J_V}( \bar{X}, \bar{Y} )$ is a coboundary. As a matrix, $\psi'$ is obtained from $\psi$ by inflation. This reduces the problem of determing whether or not $\psi$ is null-homotopic to the subspace membership problem for the pair $(B, \psi')$. If $k$ is a field this may be computed by a row-reduction, and otherwise using Gr\"obner bases.

\begin{example} Suppose $V = z^N$ in $k[z]$ for $N \ge 2$. Take matrix factorisations $X,Y$ with the same underlying free $R = k[z]$-module $X = Y = R \oplus R \theta$ and differentials
\[
d_X = z^i \theta + z^{N-i} \theta^* = \begin{pmatrix} 0 & z^{N-i} \\ z^i & 0 \end{pmatrix}, \qquad d_Y = z^j \theta + z^{N-j} \theta^* = \begin{pmatrix} 0 & z^{N-j}\\ z^j & 0 \end{pmatrix}
\]
where $1 \le i,j \le N/2$. Using the basis $\Hom_R(X,Y) = R \theta \theta^* \oplus R \theta^* \theta \oplus R\theta \oplus R \theta^*$,
\begin{equation}\label{eq:differential_hom}
d_{\Hom} = \begin{pmatrix} 0 & 0 & z^{N-i} & z^j \\ 
0 & 0 & z^{N-j} & z^i \\
-z^i & z^j & 0 & 0 \\
z^{N-j} & -z^{N-i} & 0 & 0 \end{pmatrix}\,.
\end{equation}
Let us now compute the cut $X^{\vee} \l Y$ with its Clifford action. With
\[
J_V = k[z]/z^{N-1} = k \cdot 1 \oplus k \cdot z \oplus \cdots k \cdot z^{N-2}
\]
the matrix of multiplication by $z$ on $J_V$ is $[z] = \begin{pmatrix} 0 & 0\\ I_{N-2} & 0 \end{pmatrix}$. The differential on
\[
X^{\vee} \l Y = \Hom_{J_V}( \bar{X}, \bar{Y} ) = J_V \theta \theta^* \oplus J_V \theta^* \theta \oplus J_V\theta \oplus J_V \theta^*
\]
is then obtained from \eqref{eq:differential_hom} by replacing every $z$ by the matrix $[z]$.

The Clifford algebra $A_1$ acts by two generators $a, a^\dagger$. To determine their matrices, note that with $t = z^{N-1}$, $k[z]$ is a free $k[t]$-module with basis $1, \ldots, z^{N-2}$. Given a polynomial $f(z)$ we may uniquely write it as $f(z) = \sum_{l = 0}^{N-2} f_l \cdot z^l$ with $f_l \in k[t]$. There is a connection $\nabla: k[z] \lto k[z] \otimes_{k[t]} \Omega^1_{k[t]/k}$ whose associated $k$-linear operator $\partial_t$ is
\[
\partial_t: k[z] \lto k[z], \qquad \partial_t(f) = \sum_{l = 0}^{N-2} \frac{\partial}{\partial t}( f_l ) z^l \,.
\]
Notice that for $0 \le q \le N - 1$, $\partial_t(z^q) = \delta_{q (N-1)} \cdot 1$.

The Atiyah class is the $k[t]$-linear operator $\At = [d_{\Hom}, \partial_t]$ on $\Hom_R(X,Y)$. We may compute it for example on generators by
\[
\At( \theta \theta^* ) = \partial_t d_{\Hom}( \theta \theta^* ) = - \partial_t( z^i ) \theta + \partial_t( z^{N-j} ) \theta^* = - \delta_{i (N-1)} \theta + \delta_{(N-j)(N-1)} \theta^*\,.
\]
The Atiyah class passes to a $k$-linear operator $a^\dagger$ on $X^{\vee} \l Y$ with matrix ($I$ denotes $I_{N-1}$)
\begin{equation}\label{eq:aityah_class_hom_example}
a^\dagger = \begin{pmatrix} 0 & 0 & \delta_{i1} \cdot I & \delta_{j(N-1)} \cdot I \\ 
0 & 0 & \delta_{j1} \cdot I & \delta_{i(N-1)} \cdot I \\
- \delta_{i(N-1)} \cdot I & \delta_{j(N-1)} \cdot I & 0 & 0 \\
\delta_{j1} \cdot I & - \delta_{i1} \cdot I & 0 & 0 \end{pmatrix}\,.
\end{equation}
Similarly one can determine that the matrix of $\lambda$ on $X^{\vee} \l Y$ is
\[
\lambda = \begin{pmatrix} 0 & 0 & 0 & \frac{j}{N} [z^{j-1}] \\
0 & 0 & \frac{N-j}{N} [z^{N-j-1}] & 0\\
0 & \frac{j}{N} [z^{j-1}] & 0 & 0\\
\frac{N-j}{N} [z^{N-j-1}] & 0 & 0 & 0\end{pmatrix}\,.
\]
By definition $a = - \lambda - \lambda^2 a^\dagger$ so this completes the description of the cut $( X^{\vee} \l Y, \{ a, a^\dagger \} )$. Splittng the idempotent $a^\dagger a$ on the $\mathbb{Z}_2$-graded complex $X^{\vee} \l Y$ of finite rank free $k$-modules yields a finite model for $\Hom_R(X,Y)$.
\end{example}

\appendix

\section{Homological perturbation and fixed points}\label{section:homolog_fix}

A fundamental role in this paper is played by the homological perturbation lemma. While this result is usually stated for complexes the standard results generalise immediately to linear factorisations, and in this section we collect these standard results. But we begin by recalling the motivating problem for which the perturbation lemma is the solution.

Let $Q$ be a complex of infinite rank free abelian groups. Suppose that the cohomology is finitely generated, so that $Q$ contains in some sense only a ``finite'' amount of information. One can ask for a description which makes this finiteness manifest: the obvious answer is to simply list the cohomology groups of $Q$, but a more flexible and categorical solution is to find a finite representative in the homotopy equivalence class of $Q$.

A \emph{deformation retract} of complexes is a pair of morphisms
\begin{equation}\label{eq:example_defo_retract}
\xymatrix@C+2pc{
C \ar@<-1ex>[r]_\nabla & Q, \ar@<-1ex>[l]_f
} \quad \phi
\end{equation}
and a degree $-1$ operator $\phi: Q \lto Q$ satisfying $f \nabla = 1_C, \nabla f = 1_Q - [d_Q, \phi]$ so that $f, \nabla$ are mutually inverse homotopy equivalences. A deformation retract is \emph{strong} if it satisfies some additional conditions, listed below. The point is that if $C$ is finitely generated, then \eqref{eq:example_defo_retract} gives the desired finite representative in the homotopy equivalence class of $Q$. 

\begin{example}\label{example:koszulsplit} Consider the following diagram of complexes of $k$-modules
\[
\xymatrix@C+2pc@R+1pc{
0 \ar[d] \ar@<-1ex>[r]_-{\nabla} & k[x] \ar@<-1ex>[l]_-{f} \ar[d]^-{x}\\
k \ar@<-1ex>[r]_-{\nabla} & k[x] \ar@<-1ex>[l]_-{f} \ar@/_3pc/@{.>}[u]_-{\phi}
}
\]
in which the first column is the complex $C$ which has $k$ concentrated in degree zero, and the second column $Q$ is the Koszul complex on $x$. The map $f: k[x] \lto k$ sends every polynomial to its constant term, while $\nabla: k \lto k[x]$ is the inclusion of the constants. The operator $\phi$ is defined by $\phi( x^n ) = x^{n-1}$ and $\phi(1) = 0$. Together these maps are a deformation retract giving $k$ as the ``finite model'' of $Q$.
\end{example}

Suppose now that a deformation retract computing a finite representative for $Q$ cannot be found immediately, but that we know how to decompose the differential as $d_Q = d + \tau$ in such a way that there is a finite representative for the complex $(Q,d)$. The perturbation lemma allows us, with some conditions, to ``mix in'' the operator $\tau$ to obtain a finite model for the original complex $Q$.
\\

Let $R$ be a commutative ring and $W \in R$. Everything we say applies, in the case where $W = 0$, to both $\mathbb{Z}_2$-graded and $\mathbb{Z}$-graded complexes.  For our purposes the reformulation of the perturbation lemma in terms of fixed points by Barnes and Lambe is more natural, so we emphasise the ``splitting homotopies'' of \cite{barneslambe}.

\begin{definition} A \emph{splitting homotopy} on a linear factorisation $(A,d)$ of $W$ is a degree $-1$ operator $\phi$ on $A$ satisfying
\begin{itemize}
\item[(i)] $\phi^2 = 0$,
\item[(ii)] $\phi d \phi = \phi$.
\end{itemize}
A \emph{morphism} $(A,d, \phi) \lto (A,d',\phi')$ of splitting homotopies is a morphism $\alpha: A \lto A'$ of linear factorisations satisfying $\phi' \alpha = \alpha \phi$.
\end{definition}

The condition (ii) says that $\phi$ is a \emph{fixed point} of the operator $F(x) = x d x$.

\begin{definition} A \emph{strong deformation retract} of linear factorisations of $W$ is a pair of morphisms of linear factorisations
\begin{equation}\label{eq:defn_sdr}
\xymatrix@C+2pc{
(M,d) \ar@<-1ex>[r]_\nabla & (A,d), \ar@<-1ex>[l]_f
} \quad \phi
\end{equation}
together with a degree $-1$ operator $\phi: A \lto A$ satisfying
\begin{itemize}
\item[(i)] $f \nabla = 1_M$,
\item[(ii)] $\nabla f = 1_A - [d, \phi]$,
\item[(iii)] $\phi^2 = 0$,
\item[(iv)] $\phi \nabla = 0$,
\item[(v)] $f \phi = 0$.
\end{itemize}
A \emph{morphism} of strong deformation retracts is a commutative diagram
\[
\xymatrix@C+2pc{
(M,d) \ar[d]_{\alpha} \ar@<-1ex>[r]_\nabla & (A,d)\ar[d]^\beta \ar@<-1ex>[l]_f\\
(M',d') \ar@<-1ex>[r]_{\nabla'} & (A',d') \ar@<-1ex>[l]_{f'}
}
\]
that is, a pair of morphisms $\alpha, \beta$ such that $\beta \nabla = \nabla' \alpha$, $\alpha f = f' \beta$ and $\phi' \beta = \beta \phi$.
\end{definition}

We follow the conventions of \cite{barneslambe} in defining strong deformation retracts; up to a sign this the same as the special deformation retracts of \cite{crainic} and \cite{dm1102.2957}. Namely, a strong deformation retract in the above sense is the same as a special deformation retract in the sense of \emph{ibid.} with $h = - \phi$.

\begin{lemma}\label{lemma:equivocate} There is an equivalence between the category of splitting homotopies $\cat{C}_1$ and the category of  strong deformation retracts $\cat{C}_2$.
\end{lemma}
\begin{proof}
We briefly recall the construction from \cite[p.883]{barneslambe}, which is not stated in terms of categories but is obviously functorial. Given a strong deformation retract \eqref{eq:defn_sdr} the data $(A, \phi)$ is a splitting homotopy. In the reverse direction, if $(A, \phi)$ is a splitting homotopy then $\pi = 1_A - [d, \phi]$ is idempotent and we define the linear factorisation $M = \operatorname{im}(\phi)$ with the associated projection $f$ and inclusion $\nabla$. Then this is a strong deformation retract, together with the original $\phi$.
\end{proof}

%Taking splitting homotopies as the primary object has the advantage of hiding some of the incidental complexity in the perturbation lemma.
Here is the problem that the formalism is designed to solve: suppose $(A,d)$ is a linear factorisation of $W$ and that $\tau$ is a degree $+1$ operator on $A$ such that $(A, d + \tau)$ is a linear factorisation of $V$ (possibly different to $W$). This $\tau$ is called the \emph{perturbation}.

\begin{definition} Given a splitting homotopy $\phi$ on $(A,d)$ the \emph{transference problem} is to find a splitting homotopy $\phi'$ on $(A, d + \tau)$ such that $\im(\pi) \cong \im(\pi')$ as graded $R$-modules, where $\pi = 1_A - [d, \phi]$ and $\pi' = 1_A - [d + \tau, \phi']$. 
\end{definition}

That is, with $\xi = d + \tau$ the problem is to find a fixed point of the operator
\[
F(x) = x \xi x
\]
among operators with $x^2 = 0$ satisfying the boundary condition $\im(1_A - [d,x]) \cong \im(\pi)$. If $\phi \tau$ has finite order, i.e. $(\phi \tau)^m = 0$ for some $m$, then
\[
\phi_\infty = \sum_{m \ge 0} (-1)^m (\phi \tau)^m \phi = \phi - \phi \tau \phi + \phi \tau \phi \tau \phi - \cdots
\]
is a solution of this fixed point problem:

\begin{theorem}[Perturbation lemma]\label{theorem:pertlemma} Suppose $\phi \tau$ has finite order. Then $\phi_\infty$ is a splitting homotopy on the linear factorisation $(A, \xi)$ and satisfies the isomorphism condition in the transference problem.
\end{theorem}
\begin{proof}
The proof for complexes \cite[p.886]{barneslambe} goes through unchanged for linear factorisations.
\end{proof}

The more common statement of the basic perturbation lemma involves the deformation retract corresponding to the splitting homotopy $\phi_\infty$ under the equivalence of Lemma \ref{lemma:equivocate}. The statement is that given a splitting homotopy $\phi$ as above with associated deformation retract \eqref{eq:defn_sdr} there is a strong deformation retract
\begin{equation}\label{eq:perturbedsdr}
\xymatrix@C+2pc{
(M,d_\infty) \ar@<-1ex>[r]_{\nabla_\infty} & (A, \xi), \ar@<-1ex>[l]_{f_\infty}
} \quad \phi_\infty
\end{equation}
where $A = \tau( 1 + \phi \tau )^{-1} = \sum_{m \ge 0} (-1)^m \tau (\phi \tau)^m$ and
\begin{align*}
\nabla_\infty &= \nabla - \phi A \nabla,\\
f_\infty &= f - f A \phi,\\
d_\infty &= d + f A \nabla\,.
\end{align*}
As explained in \cite{barneslambe}, to derive these formulas one has only to notice that the sub-complex in the deformation retract associated to $\phi_\infty$ can, by the isomorphism
\[
\im(1 - [\xi, \phi_\infty]) \cong \im(\pi) = M
\]
be identified with $M$, and the induced maps are the $\nabla_\infty, f_\infty, d_\infty$ given above. It is also possible to give a direct proof as in \cite{crainic}. To see that $\phi_\infty = (1 + \phi \tau)^{-1} \phi$ is the universal solution in the input data $\phi, d$ and $\tau$, one just studies the fixed point equation $F(x) = x \xi x$ and develops a recurrence relation. The standard way of generating fixed points, say by iterating $F$ infinitely many times on $\phi$, produces the same answer provided one takes the limit of the coefficients in $\mathbb{Q}$ with respect to the $2$-adic valuation; see \cite[p.885]{barneslambe}.

\section{Tensor products in supercategories}\label{section:tensorproduct_supcat}

Let $\cat{C}$ be an idempotent complete supercategory.

\begin{definition} Let $V$ be a $\mathbb{Z}_2$-graded $k$-module and $X$ an object of $\cat{C}$. An object of $\cat{C}$ representing the functor $\Hom^0_k( V, \cat{C}^*(X, -))$ is denoted $V \otimes_k X$ if it exists. That is to say, the tensor product consists of an object $V \otimes_k X$ and a natural isomorphism
\[
\rho_Y: \cat{C}( V \otimes_k X, Y ) \lto \Hom^0_k( V, \cat{C}^*(X,Y) )\,.
\]
Such a pair is unique up to unique isomorphism, if it exists.
\end{definition}

\begin{remark} 
From $\rho$ we also obtain an isomorphism of $\mathbb{Z}_2$-graded $k$-modules
\[
\cat{C}^*( V \otimes_k X, Y ) \cong \Hom_k^*( V, \cat{C}^*(X,Y) )\,.
\]
The tensor product $V \otimes_k X$ is made functorial in $V$ and $X$ in such a way that $\rho$ is natural in all its variables.
\end{remark}

\begin{example} If $V = k^{\oplus n} \oplus k[1]^{\oplus m}$ is finite and free then
\[
V \otimes_k X = X^{\oplus n} \oplus \Psi X ^{\oplus m}
\]
is a representing object, with the obvious isomorphism $\rho$.
\end{example}

\begin{lemma} If $\cat{C}$ is idempotent complete and $V$ is a finitely generated projective $\mathbb{Z}_2$-graded $k$-module, then $V \otimes_k X$ exists for any object $X$.
\end{lemma}
\begin{proof}
Let $V'$ be a finite free $k$-module with $f: V' \lto V, g: V \lto V'$ satisfying $fg = 1$. This determines an idempotent $1 \otimes e$ on $V' \otimes_k X$, which splits by hypothesis in $\cat{C}$. The object splitting this idempotent has the right property to be $V \otimes_k X$.
\end{proof}
% via morphisms $F,G$ with
%\[
%\xymatrix@C+1pc{
%Q \ar@<0.8ex>[r]^-{G} & V' \otimes_k X \ar@<0.8ex>[l]^-{F} 
%} \qquad FG = 1, GF = 1 \otimes e
%\]
%for some object $Q$. We have
%\begin{align*}
%\cat{C}( Q, Y ) &\cong \{ f: V' \otimes_k X \lto Y \l f(1 \otimes e) = 1 \otimes e \}\\
%&\cong \{ \hat{f} \in \Hom_k( V', \cat{C}^*(X,Y)) \l \hat{f} e = e \}\\
%&\cong \Hom_k( V, \cat{C}^*(X,Y))\,.
%\end{align*}
%and hence $V \otimes_k X = Q$ exists.
%\end{proof}

More generally, let $A$ be a $\mathbb{Z}_2$-graded $k$-algebra.

\begin{definition} Given a right $\mathbb{Z}_2$-graded $A$-module $V$ and an object $X$ of $\cat{C}$, we denote by $V \otimes_A X$ the object representing the functor $\Hom^0_A( V, \cat{C}^*(X,-))$ if it exists. That is, the tensor product is an object $V \otimes_A X$ together with a natural isomorphism
\begin{equation}\label{eq:isodefatensor}
\rho_Y: \cat{C}(V \otimes_A X, Y) \lto \Hom^0_A( V, \cat{C}^*(X,Y))\,.
\end{equation}
As before, the tensor product is functorial in both $V$ and $X$.
\end{definition}

Henceforth we concentrate our attention on $\mathbb{Z}_2$-graded $k$-algebras $A$ which are \emph{Morita trivial} in the sense that they are isomorphic to an algebra of the form $\End_k(P)$ for a finite rank free $\mathbb{Z}_2$-graded $k$-module $P$.

\begin{lemma} If $A \cong \End_k(P)$ then for any $A$-module $X$ in $\cat{C}$ there is an object $\widetilde{X}$ and an isomorphism of $A$-modules $X \cong P \otimes_k \widetilde{X}$.
\end{lemma}
\begin{proof}
The natural idempotents in $A$ act as idempotents on $X$, which split.
\end{proof}

\begin{lemma} If $A \cong \End_k(P)$ and $V$ is a $\mathbb{Z}_2$-graded right $A$-module which is finitely generated and projective as a $k$-module, then $V \otimes_A X$ exists for any $A$-module $X$.
\end{lemma}
\begin{proof}
Write $P^* = \Hom_k(P,k)$ so that $V \cong \widetilde{V} \otimes_k P^*$ as right $A$-modules for some $\mathbb{Z}_2$-graded $k$-module $\widetilde{V}$. Since $V$ is finitely generated projective over $k$, so is $\widetilde{V}$. By the previous lemma, $X \cong P \otimes_k \widetilde{X}$ for some object $\widetilde{X}$, so we might expect that
\[
V \otimes_A X \cong (\widetilde{V} \otimes_k P^*) \otimes_A (P \otimes_k \widetilde{X}) \cong \widetilde{V} \otimes_k \widetilde{X}
\]
Working backwards: $\widetilde{V} \otimes_k \widetilde{X}$ exists and has the right universal property since
\begin{align*}
\cat{C}(\widetilde{V} \otimes_k \widetilde{X}, Y) &\cong \Hom_k( V \otimes_A P, \cat{C}^*(\widetilde{X}, Y))\\
&\cong \Hom_A( V, P^* \otimes_k \cat{C}^*(\widetilde{X}, Y))\\
&\cong \Hom_A( V, \cat{C}^*( X, Y ) )
\end{align*}
as claimed.
\end{proof}

From now on $A,B,C$ are Morita trivial $k$-algebras. Recall $\cat{C}_A$ denotes the category of left $A$-modules in $\cat{C}$.

\begin{definition}\label{defn:psimodule} Given a left $A$-module $X$ in $\cat{C}$, the object $\Psi X$ is a left $A$-module, where $a \in A_0$ acts by $\Psi(a)$ and $a \in A_1$ acts by
\[
\xymatrix{
\Psi (X) \ar[r]^-{\Psi(a)} & \Psi \Psi (X) \ar[r]^-{-1} & \Psi \Psi (X)
}\,.
\]
This determines a functor $\Psi: \cat{C}_A \lto \cat{C}_A$.
\end{definition}

\begin{lemma} $(\cat{C}_A, \Psi)$ is a supercategory.
\end{lemma}
\begin{proof}
The natural isomorphism $\xi_X: \Psi \Psi (X) \lto X$ is given by the identity in $\cat{C}$.
\end{proof}

Let $V$ be a $B$-$A$-bimodule which is finitely generated and projective over $k$. If $X$ is a left $A$-module in $\cat{C}$ then $V \otimes_A X$ is made into a left $B$-module in $\cat{C}$ uniquely such that \eqref{eq:isodefatensor} is a natural isomorphism of $B$-modules. Hence the tensor product gives a functor
\[
\Phi_V = V \otimes_A (-): \cat{C}_A \lto \cat{C}_B\,.
\]
Given a $B$-$A$-bimodule $V$ and an object $X$ for which $V \otimes_A X$ exists, there are natural isomorphisms of $B$-modules
\[
\Psi V \otimes_A X \cong \Psi(V \otimes_A X) \cong V \otimes_A \Psi X
\]
induced from the isomorphisms
\[
\Hom_A( \Psi V, \cat{C}^*(X, -)) \cong \Hom_A( V, \Psi \cat{C}^*(X, -) ) \cong \Hom_A( V, \cat{C}^*(\Psi X, -))\,.
\]
It follows that $\Phi_V$ is a superfunctor.

\begin{lemma}\label{lemma:assoctensor} Let $W$ be a $C$-$B$-bimodule and $V$ a $B$-$A$-bimodule. Then there is a natural isomorphism of $C$-modules
\[
(W \otimes_B V) \otimes_A X \cong W \otimes_B ( V \otimes_A X )
\]
\end{lemma}
\begin{proof}
This follows from the isomorphism
\begin{align*}
\Hom_A(W, \cat{C}^*(V \otimes_A X, -)) &\cong \Hom_A( W, \Hom_A^*(V, \cat{C}^*(X, -)))\\
&\cong \Hom_A(W \otimes_A V, \cat{C}^*(X, -))\,.
\end{align*}
\end{proof}

Now we specialise to the Clifford algebras $A_n$ and their modules.

\begin{lemma}\label{lemma:simplehom} Let $X$ be a $\mathbb{Z}_2$-graded left $A_n$-module. There is a $k$-linear isomorphism
\[
\Hom_{A_n}^0( S_n, X ) \cong \{ x \in X^0 \l a_i^\dagger \cdot x = 0 \text{ for all } 1 \le i \le n \}
\]
defined by $f \mapsto f(1)$. If $X$ is a right $A_n$-module there is a $k$-linear isomorphism
\[
\Hom^0_{A_n}( (S_n)^*, X ) \cong \{ x \in X^0 \l x \cdot a_i = 0 \text{ for all } 1 \le i \le n \}
\]
defined by $f \mapsto f(1^*)$.
\end{lemma}

\begin{lemma}\label{lemma:morphism_out_1} For an $A_n$-module $Y$ there is a $k$-linear isomorphism
\[
\cat{C}_{A_n}( S_n \otimes_k X, Y ) \cong \{ \delta: X \lto Y \l a_i^\dagger \circ \delta = 0 \text{ for all } 1 \le i \le n \}
\]
defined by sending $\alpha: S_n \otimes_k X \lto Y$ to the composite
\[
\xymatrix@C+2pc{
X \cong k \otimes_k X \ar[r]^-{\iota \otimes 1} & S_n \otimes_k X \ar[r]^-{\alpha} & Y
}
\]
where $\iota: k \lto S_n$ is defined by $\iota(1) = 1$.
\end{lemma}
\begin{proof}
This follows from Lemma \ref{lemma:simplehom} and the isomorphism
\[
\cat{C}_{A_n}(S_n \otimes_k X, Y ) \cong \Hom_{A_n}( S_n, \cat{C}^*(X,Y) )\,.
\]
\end{proof}

More generally

\begin{lemma}\label{lemma:morphisms_two_forms} Let $X \in C_{A_p}$ and $Y \in \cat{C}_{A_q}$. There is a $k$-linear isomorphism
\begin{align*}
\cat{C}_{A_q}( S_{q,p} \otimes_{A_p} X, Y ) \cong \{ \delta: X \lto Y \l \delta \circ a_i &= 0 \text{ for } 1 \le i \le p \text{ and }\\ a_i^\dagger \circ \delta &= 0 \text{ for } 1 \le i \le q \}
\end{align*}
defined by sending $\alpha: S_{q,p} \otimes_{A_p} X \lto Y$ to the composite
\[
\xymatrix@C+2pc{
X \cong A_p \otimes_{A_p} X \ar[r]^-{\iota \otimes 1} & S_{q,p} \otimes_{A_p} X \ar[r]^-{\alpha} & Y
}
\]
where $\iota: A_p \lto S_{q,p}$ is defined by $\iota(1) = 1 \otimes 1^*$.
\end{lemma}
\begin{proof}
Using the previous lemma and Lemma \ref{lemma:simplehom},
\begin{align*}
\cat{C}_{A_q}( S_{q,p} \otimes_{A_p} X, Y ) &\cong \Hom^0_{A_p-A_q}( S_{q} \otimes_k S_p^* , \cat{C}^*(X,Y))\\
&\cong \Hom^0_{A_q}( S_q, \Hom^*_{A_p}( S_p^*, \cat{C}^*(X,Y) ))\\
&\cong \{ \delta \in \Hom^0_{A_q}( S_p^*, \cat{C}^*(X,Y) ) \l a_i^\dagger \circ \delta = 0 \text{ for } 1 \le i \le q \}\\
&\cong \{ \delta: X \lto Y \l \delta \circ a_i = 0 \text{ for } 1 \le i \le p \text{ and } a_i^\dagger \circ \delta = 0 \text{ for } 1 \le i \le q \}
\end{align*}
as claimed.
\end{proof}

\section{Extending cut functors}\label{app:proof_cut_extension}

The purpose of this appendix is to give the proof of Proposition \ref{prop:extend_cut_functor}. First, note that

\begin{lemma}\label{lemma:altcomppersp} Let $\cat{C}$ be a small idempotent complete supercategory and take morphisms
\[
\xymatrix{
(X,n) \ar[r]^-{\alpha} & (Y,m) \ar[r]^-{\beta} & (Z,l)
}\\
\]
in $\cat{C}^\bullet$. Under \eqref{eq:isodefatensor} these correspond to bimodule maps
\[
\widetilde{\alpha}: S_{m,n} \lto \cat{C}^*(X,Y)\,, \qquad \widetilde{\beta}: S_{l,m} \lto \cat{C}^*(Y,Z)\,.
\]
The composite $\beta \circ \alpha: S_{l,n} \otimes_{A_n} X \lto Z$ in $\cat{C}^\bullet$ corresponds to the composite
\[
\xymatrix{
S_{l,n} \cong S_{l,m} \otimes_{A_m} S_{m,n} \ar[r]^-{\widetilde{\beta} \otimes \widetilde{\alpha}} & \cat{C}^*(Y,Z) \otimes_{A_m} \cat{C}^*(X,Y) \ar[r]^-{\circ} & \cat{C}^*(X,Z)
}\,.
\]
\end{lemma}
\begin{proof}
Follows by evaluation on $1 \otimes 1^* \in S_{l,n}$.
\end{proof}

\begin{proof}[Proof of Proposition \ref{prop:extend_cut_functor}]
Given $(X,n) \in \cat{C}_1^\bullet$ and $(Y,m) \in \cat{C}_2^\bullet$ we have $T(X,Y) = (Z,t)$ for some object $Z$ of $\cat{C}_3$. Given $a \in A_n$, say of odd degree, there is a morphism of $A_t$-modules $Z \lto \Psi Z$ given by
\[
\xymatrix@C+2pc{
T(X,Y) \ar[r]^-{T(a \otimes 1)} & T(\Psi X, Y ) \ar[r]^-{\tau} & \Psi T(X,Y)
}
\]
and this defines an algebra morphism $A_n \lto \cat{C}_3^*(Z,Z)$. Doing the same for $A_m$ defines a morphism $A_{m+t+n} \cong A_m \otimes A_t \otimes A_n \lto \cat{C}_3^*(Z,Z)$ and an object $(Z, m + t +n) \in \cat{C}_3^\bullet$. We define $\widetilde{T}$ on objects by
\begin{equation}\label{eq:ttilde}
\widetilde{T}( (X,n), (Y,m) ) = (Z, m + t + n)\,.
\end{equation}
Next we define $\widetilde{T}$ on morphisms. Given $\alpha: (X_1,n_1) \lto (X_2,n_2)$ in $\cat{C}_1^\bullet$ and $T(X_i,Y) = (Z_i,t)$ for $i = 1,2$ we must construct a morphism
\[
\widetilde{T}(\alpha,1): (Z_1, m+t+n_1) \lto (Z_2, m+t+n_2)\,.
\]
Since $\alpha$ corresponds to an $A_{n_1}$-linear map $\widetilde{\alpha}: V \lto \cat{C}_1^*(X_1,X_2)$ we can compose with $T$ to get an $A_{n_2}$-$A_{n_1}$-bilinear map $\eta := T(1,Y) \circ \widetilde{\alpha}: V \lto \cat{C}_3^*(Z_1,Z_2)$. This extends to a morphism of $A_{n_2+c}$-$A_{n_1+c}$-bimodules
\[
V \otimes_k A_c \lto \cat{C}_3^*(Z_1,Z_2), \qquad v \otimes a \mapsto \eta(v) a\,.
\]
There is a canonical isomorphism of $A_{n+2+c}$-$A_{n_1+c}$-bimodules
\begin{align*}
S_{n_2+c,n_1+c} \cong S_{n_2+c} \otimes_k S_{n_1+c}^* \cong (S_{n_2} \otimes_k S_c) \otimes_k (S_c^* \otimes_k S_{n_1}^*) \cong V \otimes_k (S_c \otimes_k S_c^*) \cong V \otimes_k A_c
\end{align*}
so we have constructed $S_{n_2+c,n_1+c} \lto \cat{C}_3^*(Z_1,Z_2)$ which corresponds to a morphism of $A_{n_2+c}$-modules $S_{n_2+c, n_1+c} \otimes_{A_{n_1+c}} Z_1 \lto Z_2$ which we define to be $\widetilde{T}(\alpha,1)$. Functoriality of this construction follows from Lemma \ref{lemma:altcomppersp}. The functoriality in the other variable follows similarly. By construction $\widetilde{T}$ makes both of the required diagrams commute.
\end{proof}

\section{Constructing superbicategories}\label{section:constructing_superbicategories}

This appendix collects the data needed for constructing a superbicategory $\cat{B}$ from a collection of supercategories $\cat{B}(a,b)$. Given a bicategory $\cat{B}$ and for each pair of objects $a,b$ the structure of a supercategory on $\cat{B}(a,b)$, we denote composition by
\[
T: \cat{B}(a, b) \otimes_k \cat{B}(b,c) \lto \cat{B}(a,c)\,.
\]
We assume this is a superfunctor, i.e. that \eqref{eq:psicomp1}-\eqref{eq:psicomp3} commute. Since our main example is constructing the superbicategory associated to a cut system, we write $T(X,Y)$ as $Y \l X$. We denote the associator by $\alpha$, the units in $\cat{B}$ by $\Delta_a: a \lto a$, and the unitors for $X: a \lto b$ by $\lambda_X: \Delta_b \l X \lto X$ and $\rho_X: X \l \Delta_a \lto X$.

Suppose that for all objects $a,b,c$ we are given natural isomorphisms
\begin{align*}
T \circ ( \Psi \otimes 1 ) \lto \Psi \circ T, \qquad T \circ ( 1 \otimes \Psi ) \lto \Psi \circ T
\end{align*}
both of which will be denoted $\tau$. We define for an object $a$ the $1$-morphism $\Psi_a = \Psi( \Delta_a )$ and the $2$-isomorphism $\xi_a$ to be the following composite (using the structure maps of $\cat{B}$ and isomorphisms $\tau$)
\[
\xi_a: \Psi_a \l \Psi_a = \Psi( \Delta_a ) \l \Psi( \Delta_a ) \cong \Psi( \Delta_a \l \Psi(\Delta_a ) ) \cong \Psi^2( \Delta_a \l \Delta_a ) \cong \Delta_a \l \Delta_a \cong \Delta_a\,.
\]
Similarly given a $1$-morphism $X: a \lto b$ we have a $2$-isomorphism $\gamma_X$
\[
\gamma_X: X \l \Psi_a = X \l \Psi( \Delta_a ) \cong \Psi( X \l \Delta_a ) \cong \Psi(X) \cong \Psi( \Delta_b \l X ) \cong \Psi( \Lambda_b ) \l X = \Psi_b \l X\,.
\]
Given a $1$-morphism $X: a \lto b$ in $\cat{B}$ there are diagrams
\begin{gather}
\xymatrix@R+1pc@C+1pc{
\Delta_b \l \Psi (X) \ar[dr]_-{\lambda_{\Psi a}} \ar[rr]^-{\tau} & & \Psi( \Delta_b \l X ) \ar[dl]^-{\Psi( \lambda )}\\
& \Psi(X)
} \label{eq:psiunitor1}\\
\xymatrix@R+1pc@C+1pc{
\Psi(X) \l \Delta_a \ar[rr]^-{\tau}\ar[dr]_-{\rho} & & \Psi( X \l \Delta_a ) \ar[dl]^-{\Psi(\rho)}\\
& \Psi(X)
} \label{eq:psiunitor2}
\end{gather}
whose commutativity expresses compatibility of the functors $\Psi$ with the units, while for a composable triple $X,Y,Z$ the commutativity of the diagrams
\begin{gather}
\xymatrix{
(Z \l Y) \l \Psi(X) \ar[dd]_-{\tau} \ar[rr]^-{\alpha} & & Z \l ( Y \l \Psi(X)) \ar[d]^-{\tau}\\
 & & Z \l \Psi( Y \l X ) \ar[d]^-{\tau}\\
\Psi( (Z \l Y) \l X ) \ar[rr]_-{\alpha} & & \Psi( Z \l ( Y \l X ) )
} \label{eq:psithirdlastdia}\\
\xymatrix{
(Z \l \Psi(Y)) \l X \ar[d]_-{\tau} \ar[rr]^-{\alpha} & & Z \l ( \Psi(Y) \l X) \ar[d]^-{\tau}\\
\Psi( Z \l Y ) \l X \ar[d]_-{\tau} & & Z \l \Psi( Y \l X ) \ar[d]^-{\tau}\\
\Psi( (Z \l Y) \l X ) \ar[rr]_-{\alpha} & & \Psi( Z \l ( Y \l X ) )
} \label{eq:psimiddledia}\\
\xymatrix{
(\Psi(Z) \l Y) \l X \ar[d]_-{\tau} \ar[rr]^-{\alpha} & & \Psi(Z) \l ( Y \l X ) \ar[dd]^-{\tau}\\
\Psi(Z \l Y) \l X \ar[d]_-{\tau} & & \\
\Psi( (Z \l Y) \l X ) \ar[rr]_-{\alpha} & & \Psi( Z \l ( Y \l X ) )
}\label{eq:psilastdia}
\end{gather}
expresses compatibility of $\Psi$ with the associator of $\cat{B}$.

\begin{lemma}\label{lemma:constructingsuper}
Given a bicategory $\cat{B}$ with the structure of a supercategory on $\cat{B}(a,b)$ for all pairs $a,b$, and natural isomorphisms $\tau$, suppose that \eqref{eq:psiunitor1}-\eqref{eq:psilastdia} commute. Then with $\Psi, \xi$ and $\gamma$ defined as above, $\cat{B}$ is a superbicategory.
\end{lemma}
\begin{proof}
The proof is an easy but somewhat lengthy exercise, which we omit.
\end{proof}

\section{Clifford actions and Koszul complexes}\label{section:cliffordactkos}
% spkos series

Let $R$ be a $\mathbb{Q}$-algebra and $(X,d)$ a linear factorisation of some $W \in R$. Let $t_1,\ldots,t_n$ be a sequence of elements of $R$ acting null-homotopically on $X$ and choose degree $-1$ operators $\lambda_i$ on $X$ with $[\lambda_i, d] = t_i \cdot 1_X$.

Letting $\theta_i$ be a formal variable of degree $-1$, we define the Koszul complex $(K,d_K)$ with underlying graded module the exterior algebra $K$ on the $\theta_i$, as in \eqref{defn:koszul}. The tensor product $K \otimes X = (K \otimes X, d + d_K)$ is another linear factorisation of $W$ and on the underlying graded module $K \otimes X$ we have operators $\theta_k^* = \theta_k^* \otimes 1$ and $\theta_k = \theta_k \otimes 1$. We can define additional operators in terms of these by $\psi_k^* = \theta_k^*$ and
\begin{equation}
\psi_k = \theta_k - \sum_{m \ge 0} \sum_{q_1,\ldots,q_m} \frac{1}{(m+1)!} [ \lambda_{q_m}, [ \lambda_{q_{m-1}}, [ \cdots \big[ \lambda_{q_1}, \lambda_k ] \cdots ] \theta_{q_1}^* \cdots \theta_{q_m}^*\,.
\end{equation}
The appendix is taken up with the proof of:

\begin{theorem}\label{theorem:psik} The operators $\psi_k, \psi_k^*$ on $K \otimes X$ are closed and satisfy the Clifford relations
\begin{equation}\label{eq:clifford_reln_psi}
[ \psi_i, \psi_j ] = 0, \quad [ \psi_i^{*}, \psi_j^{*} ] = 0, \quad [ \psi_i, \psi_j^{*} ] = \delta_{ij}\,.
\end{equation}
\end{theorem}

We introduce the notation
\begin{align*}
\delta &= \sum_{i=1}^n \lambda_i \theta_i^*,\\
\exp(-\delta) &= \sum_{m \ge 0} (-1)^m \frac{1}{m!} \delta^m\,.
\end{align*}

\begin{proposition}\label{prop:equivalencekoszul} There are mutually inverse isomorphisms of linear factorisations
\[
\xymatrix@C+3pc{ (K \otimes X, d + d_K) \ar@<1ex>[r]^-{ \exp(\delta) } & (K \otimes X, d) \ar@<1ex>[l]^-{ \exp(-\delta) } }\,.
\]
\end{proposition}

Before we give the proof, we need the key observation that $[d, \delta] = \sum_i [d, \lambda_i] \theta_i^* = d_K$. More generally

\begin{lemma}\label{lemma:pert1} $[d, \delta^m] = m \delta^{m-1} d_K$ for $m \ge 1$.
\end{lemma}
\begin{proof}
\[
[ d, \delta^m ] = \sum_{i=0}^{m-1} \delta^i [d, \delta] \delta^{m-i-1} = \sum_{i=0}^{m-1} \delta^i d_K \delta^{m-i-1} = \sum_{i=0}^{m-1} \delta^i \delta^{m-i-1} d_K = m \delta^{m-1} d_K
\]
\end{proof}

\begin{lemma}\label{lemma:pert2} $[ d, \exp(-\delta) ] = - \exp(-\delta) d_K$ and $[ d, \exp(\delta) ] = \exp(\delta) d_K$.
\end{lemma}
\begin{proof}
We prove the first identity using Lemma \ref{lemma:pert1}, the second is the same:
\begin{align*}
[d, \exp(-\delta)] &= \sum_{m \ge 1} \frac{(-1)^m}{m!} [ d, \delta^m ]\\
&= \sum_{m \ge 1} \frac{(-1)^m}{m!} m \delta^{m-1} d_K\\
&= - \sum_{m \ge 0} \frac{(-1)^m}{m!} \delta^m d_K\\
&= - \exp(-\delta) d_K
\end{align*}
\end{proof}

\begin{proof}[Proof of Proposition \ref{prop:equivalencekoszul}]
It suffices to show $\exp(-\delta)$ is a morphism of linear factorisations. But using Lemma \ref{lemma:pert2}
\[
(d + d_K) \exp(-\delta) - \exp(-\delta) d = [d, \exp(-\delta)] + d_K \exp(-\delta) = 0
\]
and similarly for the other equation.
\end{proof}

The operators $\theta_i, \theta_i^*$ generate a Clifford action on the exterior algebra $K$ and therefore act via closed operators on $K \otimes X$. The $\psi_k, \psi_k^*$ are defined to be the translation of this Clifford structure to the $K \otimes X$ using the Proposition, i.e.
\begin{align}
\psi_k &= \exp(-\delta) \theta_k \exp(\delta), \label{eq:defn_psik1}\\
\psi_k^{*} &= \exp(-\delta) \theta_k^* \exp(\delta)\,. \label{eq:defn_psik2}
\end{align}

It remains to show that these operators have the form described prior to the statement of the theorem.

\begin{lemma} $\psi_k^{*} = \theta_k^*$ for $1 \le k \le n$.
\end{lemma}
\begin{proof}
Since $\delta$ commutes with $\theta_k^*$, $\exp(-\delta) \theta_k^* = \theta_k^* \exp(-\delta)$.
\end{proof}

However the operator $\theta_k$ is not closed, indeed $[ \theta_i, d + d_K ] = t_i$, and so giving a closed formula for $\psi_k^*$ is more involved. In the development below we fix an index $k$. To compute $\psi_k$ the strategy is to commute $\theta_k$ past $\exp(-\delta)$. To this end we must first compute $[\theta_k, \delta^m]$ and then $[\theta_k, \exp(-\delta)]$. The calculation begins
\begin{equation}\label{eq:catfish}
[ \theta_k, \delta^m ] = \sum_{p=0}^{m-1} \delta^p [ \theta_k, \delta ] \delta^{m-p-1} = - \sum_{p=0}^{m-1} \delta^p [ \delta, \theta_k ] \delta^{m-p-1}\,.
\end{equation}
We need to understand how to commute $\delta^p$ past the $[ \delta, \theta_k ]$. The following notation will be used for any elements of $a,b$ of a graded ring and $m \ge 1$
\[
[b, a]^{(m)} = [b, [b, [b, \cdots [b, a] \dots ]
\]
where there are $m$ occurrences of $b$. By convention $[b,a]^{(0)} = a$.

\begin{lemma} $b^m [b, a] = \sum_{i+j=m} \binom{m}{i} [b,a]^{(i+1)} b^j$ for $m \ge 1$.
\end{lemma}
\begin{proof}
By induction, with the case $m = 1$ being clear. Using the inductive hypothesis
\[
b^{m+1} [b, a] = \sum_{i+j=m} \binom{m}{i} b [b,a]^{(i+1)} b^j = \sum_{i+j=m} \binom{m}{i} \Big[ [b,a]^{(i+2)} b^j + [b,a]^{(i+1)} b^{j+1} \Big]\,.
\]
The powers of $b$ range from $0$ to $m+1$ and the coefficient of $b^j$ for $1 \le j \le m$ is
\[
\binom{m}{m-j} [b,a]^{(m-j+2)} + \binom{m}{m-j+1} [b,a]^{(m-j+2)} = \binom{m+1}{m+1-j} [b,a]^{(m-j+2)}\,.
\]
The coefficient of $b^0$ is $[b,a]^{(m+2)}$ and the coefficient of $b^{m+1}$ is $[b,a]^{(1)}$, so we are done.
\end{proof}

From the previous lemma we deduce that
\begin{equation}\label{eq:calc1}
\delta^p [ \delta, \theta_k ] = \sum_{i+j=p} \binom{p}{i} [ \delta, \theta_k ]^{(i+1)} \delta^j = \sum_{i+j=p} \binom{p}{i} [ \delta, \lambda_k ]^{(i)} \delta^j\,.
\end{equation}
The next thing to compute is $[ \delta, \lambda_k ]^{(i)}$ for any $i$. For a tuple $\bold{q} = (q_1,\ldots,q_m)$ of integers between $1$ and $n$ we write (remembering that $k$ is fixed)
\begin{gather*}
\tau_{\bold{q}} = [ \lambda_{q_m}, [ \lambda_{q_{m-1}}, \ldots [ \lambda_{q_1}, \lambda_k ] \cdots ]\\
\theta_{\bold{q}}^* = \theta_{q_1}^* \cdots \theta_{q_m}^*\,.
\end{gather*}

\begin{lemma} For $m \ge 0$
\[
[ \delta, \lambda_k ]^{(m)} = (-1)^m \sum_{q_1,\ldots,q_m} \tau_{q_1,\ldots,q_m} \theta^*_{q_1,\ldots,q_m}\,.
\]
\end{lemma}
\begin{proof}
By the inductive hypothesis
\begin{align*}
[\delta, \lambda_k]^{(m+1)} &= [ \delta, [ \delta, \lambda_k ]^{(m)} ]\\
&= (-1)^m \sum_{\bold{q}} [ \delta, \tau_{\bold{q}} \theta_{\bold{q}}^* ]\\
&= (-1)^m \sum_{\bold{q}, q_{m+1}} [ \lambda_{q_{m+1}} \theta_{q_{m+1}}^*, \tau_{\bold{q}} \theta_{\bold{q}}^* ]\\
&= (-1)^m \sum_{\bold{q}, q_{m+1}} \big( \lambda_{q_{m+1}} \theta_{q_{m+1}}^* \tau_{\bold{q}} \theta_{\bold{q}}^* + \tau_{\bold{q}} \theta_{\bold{q}}^* \lambda_{q_{m+1}} \theta_{q_{m+1}}^* \big)\\
&= (-1)^m \sum_{\bold{q}, q_{m+1}} \big( -\lambda_{q_{m+1}} \tau_{\bold{q}}  \theta_{\bold{q}}^* \theta_{q_{m+1}}^* + (-1)^m \tau_{\bold{q}} \lambda_{q_{m+1}} \theta_{\bold{q}}^* \theta_{q_{m+1}}^* \big)\\
&= (-1)^{m+1} \sum_{\bold{q}, q_{m+1}} [ \lambda_{q_{m+1}}, \tau_{\bold{q}} ] \theta_{\bold{q}}^* \theta_{q_{m+1}}^*
\end{align*}
as claimed.
\end{proof}

Continuing the computation of \eqref{eq:calc1} we have
\begin{equation}
\label{eq:calc2}
\delta^p [\delta, \theta_k ] = \sum_{i+j=p} \binom{p}{i} (-1)^i \sum_{q_1,\ldots,q_i} \tau_{q_1,\ldots,q_i} \theta^*_{q_1,\ldots,q_i} \delta^j\,.
\end{equation}

\begin{lemma}\label{lemma:commutator_wha} For $m \ge 1$
\[
[ \theta_k, \delta^m ] = \sum_{i=0}^{m-1} (-1)^{i+1} \sum_{i \le p \le m -1} \binom{p}{i} \sum_{q_1,\ldots,q_i} \tau_{q_1,\ldots,q_i} \theta_{q_1,\ldots,q_i}^* \delta^{m-i-1}
\]
\end{lemma}
\begin{proof} 
Beginning from \eqref{eq:catfish}
\begin{align*}
[ \theta_k, \delta^m ] &= - \sum_{p=0}^{m-1} \delta^p [ \delta, \theta_k ] \delta^{m-p-1}\\
&= - \sum_{p=0}^{m-1} \sum_{i+j=p} \binom{p}{i} (-1)^i \sum_{q_1,\ldots,q_i} \tau_{q_1,\ldots,q_i} \theta^*_{q_1,\ldots,q_i} \delta^j \delta^{m-p-1}
\end{align*}
\end{proof}

\begin{lemma}\label{lemma:combin} For integers $a, b \ge 0$
\begin{equation}\label{eq:combin}
\frac{b!}{(a+b+1)!} \sum_{a \le m \le a + b} \binom{m}{a} = \frac{1}{(a+1)!}
\end{equation}
\end{lemma}
\begin{proof}
Denoting the left hand side of \eqref{eq:combin} by $\Omega_{a,b}$ the proof is an induction on $b$, i.e. by computing $\Omega_{a,b+1}$ in terms of $\Omega_{a,b}$.
\end{proof}

\begin{lemma}
\[
[\theta_k, \exp(-\delta)] = \sum_{m \ge 0} \frac{1}{(m+1)!} \sum_{q_1,\ldots,q_m} \tau_{q_1,\ldots,q_m} \theta^*_{q_1,\ldots,q_m} \exp(-\delta)\,.
\]
\end{lemma}
\begin{proof}
Using Lemma \ref{lemma:combin} and Lemma \ref{lemma:commutator_wha}
\begin{align*}
[\theta_k, \exp(-\delta)] &= \sum_{m \ge 0} \frac{(-1)^m}{m!} [\theta_k, \delta^m]\\
&= \sum_{m \ge 0} \sum_{i=0}^{m-1} (-1)^{i+m+1} \frac{1}{m!} \sum_{i \le p \le m -1} \binom{p}{i} \sum_{q_1,\ldots,q_i} \tau_{q_1,\ldots,q_i} \theta_{q_1,\ldots,q_i}^* \delta^{m-i-1}\\
&= \sum_{t \ge 0} \sum_{i \ge 0} \frac{(-1)^t}{(t+1+i)!} \sum_{i \le p \le t + i} \binom{p}{i} \sum_{q_1,\ldots,q_i} \tau_{q_1,\ldots,q_i} \theta_{q_1,\ldots,q_i}^* \delta^t\\
&= \sum_{t \ge 0} \sum_{i \ge 0} \frac{(-1)^t}{t!} \frac{1}{(i+1)!} \sum_{q_1,\ldots,q_i} \tau_{q_1,\ldots,q_i} \theta^*_{q_1,\ldots,q_i} \delta^t\\
&= \sum_{m \ge 0} \frac{1}{(m+1)!} \sum_{q_1,\ldots,q_m} \tau_{q_1,\ldots,q_m} \theta^*_{q_1,\ldots,q_m} \exp(-\delta)\,.
\end{align*}
\end{proof}

\begin{proof}[Proof of Theorem \ref{theorem:psik}] 
\begin{align*}
\psi_k &= \exp(-\delta) \theta_k \exp(\delta)\\
&= \theta_k \exp(-\delta) \exp(\delta) + [ \exp(-\delta), \theta_k ] \exp(\delta)\\
&= \theta_k - [ \theta_k, \exp(-\delta) ] \exp(\delta)\\
&= \theta_k - \sum_{i \ge 0} \frac{1}{(i+1)!} \sum_{q_1,\ldots,q_i} \tau_{q_1,\ldots,q_i} \theta^*_{q_1,\ldots,q_i} \exp(-\delta) \exp(\delta)\\
&= \theta_k - \sum_{i \ge 0} \frac{1}{(i+1)!} \sum_{q_1,\ldots,q_i} \tau_{q_1,\ldots,q_i} \theta^*_{q_1,\ldots,q_i} 
\end{align*}
which completes the proof.
\end{proof}

\bibliographystyle{amsalpha}
\providecommand{\bysame}{\leavevmode\hbox to3em{\hrulefill}\thinspace}
\providecommand{\href}[2]{#2}
\begin{thebibliography}{BHLS03}

% reference for \lambda-calculus
% reference for cut
% geometry of interaction

\bibitem{barneslambe}
D.~W.~Barnes and L.~A.~Lambe, \emph{A fixed point approach to homological perturbation theory}, Proc. Amer.
  Math. Soc. \textbf{112} Number 3 (1991), 881--892.

\bibitem{bor94}
F.~Borceux, \textsl{Handbook of categorical algebra $1$}, volume $50$ of \textsl{Encyclopedia of Mathematics and its Applications}, Cambridge University Press, Cambridge, 1994.

\bibitem{buchweitz_flenner}
R.-O.~Buchweitz and H.~Flenner, \textsl{A semiregularity map for modules and applications to deformations}, Compositio Math. \textbf{137} (2003), 135--210.
    
\bibitem{ct1007.2679}
A.~{C\u ald\u araru} and S.~Willerton, \textsl{The Mukai pairing, I: a categorical approach},
New York Journal of Mathematics \textbf{16} (2010), 61--98, 
  \href{http://arxiv.org/abs/0707.2052}{[arXiv:0707.2052]}.

\bibitem{khovhompaper}
N.~Carqueville and D.~Murfet, \textsl{Computing {K}hovanov-{R}ozansky homology and defect fusion}, Algebraic \& Geometric Topology \textbf{14} (2014), 489--537, \href{http://arxiv.org/abs/1108.1081}{[arXiv:1108.1081]}. 

\bibitem{lgdual}
N.~Carqueville and D.~Murfet, \textsl{Adjunctions and defects in {L}andau-{G}inzburg models}, \href{http://arxiv.org/abs/1208.1481}{[arXiv:1208.1481]}.

\bibitem{lgdual_survey}
N.~Carqueville and D.~Murfet, \textsl{A toolkit for defect computations in Landau-Ginzburg models}, \href{http://arxiv.org/abs/1303.1389}{[arXiv:1303.1389]}, contribution to the String-Math 2012 proceedings.

\bibitem{ade}
N.~Carqueville, A.~Ros Camacho and I.~Runkel, \textsl{Orbifold equivalent potentials}, \href{http://arxiv.org/abs/1311.3354}{[arXiv:1311.3354]}.

\bibitem{cr0909.4381}
N.~Carqueville and I.~Runkel, \textsl{On the monoidal structure of matrix bi-factorisations}, J. Phys.
  A: Math. Theor. \textbf{43} (2010), 275401,
  \href{http://arxiv.org/abs/0909.4381}{[arXiv:0909.4381]}.
  
\bibitem{crainic}
M.~Crainic, \emph{On the perturbation lemma, and deformations}, \href{http://arxiv.org/abs/math/0403266}{[arXiv:0403266]}

\bibitem{d0904.4713}
T.~Dyckerhoff, \textsl{Compact generators in categories of matrix factorizations},
  Duke Math. J. \textbf{159} (2011), 223--274,
  \href{http://arxiv.org/abs/0904.4713}{[arXiv:0904.4713]}.

\bibitem{dm1102.2957}
T.~Dyckerhoff and D.~Murfet, \textsl{Pushing forward matrix factorisations}, Duke Math. J. \textbf{162} (2013), 1249--1311, \href{http://arxiv.org/abs/1102.2957}{[arXiv:1102.2957]}.

\bibitem{ellis_lauda}
A.~Ellis and A.~Lauda, \textsl{An odd categorification of $U_q(\mathfrak{sl}_2)$}, \href{http://arxiv.org/abs/1307.7816}{[arXiv:1307.7816]}.

\bibitem{enderton}
H.~B.~Enderton, \textsl{Elements of recursion theory}, Handbook of Mathematical Logic, North-Holland (1977) 527--566.

\bibitem{friedrich}
T.~Friedrich, \textsl{{D}irac operations in {R}iemannian geometry}, Graduate studies in mathematics \textbf{25}, AMS (1997).

\bibitem{gentzen}
G.~Gentzen, \textsl{The Collected Papers of Gerhard Gentzen}, (Ed. M. E. Szabo), Amsterdam, Netherlands: North-Holland, 1969.
  
\bibitem{girard_llogic}
J.-Y.~Girard, \textsl{Linear Logic}, Theoretical Computer Science 50 (1) (1987), 1--102.

\bibitem{girard_towards}
J.-Y.~Girard, \textsl{Towards a geometry of interaction}, In J.~W.~Gray and A.~Scedrov, editors, Categories in Computer Science and Logic, volume 92 of Contemporary Mathematics, 69--108, AMS (1989).

\bibitem{greuel}
G.-M.~Greuel, C.~Lossen and E.~Shustin, \textsl{Introduction to singularities and deformations}, Springer (2007).

\bibitem{kang}
S.-J.~Kang, M.~Kashiwara and S.-J.~Oh, \textsl{Supercategorification of quantum Kac-Moody algebras}, Adv. Math \textbf{242} (2013) 116--162, \href{http://arxiv.org/abs/1206.5933}{[arXiv:1206.5933]}.

\bibitem{kang2}
S.-J.~Kang, M.~Kashiwara and S.-J.~Oh, \textsl{Supercategorification of quantum Kac-Moody algebras II}, \href{http://arxiv.org/abs/1303.1916}{[arXiv:1303.1916]}.

\bibitem{Calinetal}
C.~I. Lazaroiu and D.~McNamee, unpublished.

\bibitem{McNameethesis}
D.~McNamee, \textsl{On the mathematical structure of topological defects in
  {L}andau-{G}inzburg models}, MSc Thesis, Trinity College Dublin, 2009.
  
\bibitem{cutsystems2}
D.~Murfet, \textsl{Computing with cut systems II: linear logic}, in progress.

\bibitem{selinger}
P.~Selinger, \textsl{Lecture notes on the {L}ambda calculus}, \href{http://arxiv.org/abs/0804.3434}{[arXiv:0804.3434]}.

\bibitem{vistoli}
A.~Vistoli, \textsl{Notes on {G}rothendieck topologies, fibered categories and descent theory}, \href{http://arxiv.org/abs/math/0412512}{[arXiv:0412512]}.

\bibitem{yoshino98}
Y.~Yoshino, \textsl{Tensor products of matrix factorizations}, Nagoya Math. J.
  \textbf{152} (1998), 39--56.
  
\end{thebibliography}

\end{document}